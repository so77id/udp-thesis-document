\documentclass{article}
\usepackage[utf8]{inputenc}
\usepackage[spanish]{babel}
\usepackage{natbib}
\usepackage{graphicx}





\topmargin -0.4in
\evensidemargin 0in
\oddsidemargin 0in
\textwidth 6.5in
\textheight 9.0in
\linespread{1.1}
\setlength{\parindent}{1cm}


\begin{document}
\begin{titlepage}

\begin{center}


\includegraphics[width=.8\textwidth]{./inf_tele.jpg}


\textsc{\LARGE Universidad Diego Portales}\\[1.5cm]


\textsc{\Large Ante Proyecto}\\[0.5cm]


% Title

\rule{1\textwidth}{.4pt}


{ \huge \bfseries Reconocimiento de expresiones faciales en imágenes dinámicas utilizando un descriptor basado en rayos de flujo \\[0.4cm] }


\rule{1\textwidth}{.4pt}\newline\newline\newline\newline
% Author and supervisr
\begin{minipage}{0.4\textwidth}
\begin{flushleft} \large
\emph{Autor:}\\
Miguel \textsc{Rodríguez}
\end{flushleft}
\end{minipage}
\begin{minipage}{0.4\textwidth}
\begin{flushright} \large
\emph{Profesor guía:} \\
Adin \textsc{Ramirez}\\
\emph{Comisión:} \\
Javier \textsc{Pereira}
\end{flushright}
\end{minipage}


\vfill


% Bottom of the page

{\large \today}


\end{center}

\end{titlepage}





\newpage
\section{Motivación y antecedentes}
    \subsection{Motivación}
    El reconocimiento de expresiones faciales es parte del reconocimiento de patrones, una rama de la inteligencia artificial encargada de reconocer y clasificar los patrones de comportamiento. Estudios sobre la comunicación oral indican que las expresiones faciales contribuyen con aproximadamente el 55\%  de la transmisión del mensaje, lo cual es un valor mucho mayor que la parte verbal (7\%) y la parte vocal (38\%)~\cite{commun1}. Dada la gran importancia de las expresiones en la comunicación es relevante lograr métodos eficientes tanto en la velocidad como en el reconocimiento de estas.

    \subsection{Antecedentes}
En el estudio del reconocimiento de las expresiones faciales existen una variedad de métodos que permiten realizar el reconocimiento de las expresiones canónicas con una alta tasa de aprobación.
        \begin{enumerate}
        \item \textbf{Expresiones faciales canónicas}\\
                Dado que el rostro humano tiene una infinita variedad de posibles expresiones, se establecieron 6 expresiones base o canónicas, las cuales son excluyentes entre sí, además de una expresión neutra. Las cuales son:  Surprise (Sorpresa), Happiness (Felicidad), Sadness (Tristeza), Fear (Miedo), Anger (Cólera) y Disgust (Asco).
        \item \textbf{Métodos de reconocimiento de expresiones faciales} \\
        El reconocimiento de expresiones faciales se puede dividir en dos grandes aspectos: el primero es la representación vectorial o de características de la cara, y segundo el entrenamiento y clasificaciones con ayuda de algoritmos de clasificación.
        
        \begin{enumerate}
        \item Representación vectorial o de características de la cara:\\
        Para poder utilizar los algoritmos de clasificación es necesario tener una representación de las características del objeto a clasificar. En nuestro caso para poder etiquetar imágenes o videos de rostros es necesario realizar un proceso de extracción de caracteristicas. 

En el estado del arte existen métodos o algoritmos que permiten la extracción de características, estos métodos se dividen según el tipo de entrada, las cuales pueden ser imágenes estáticas y dinámicas.  
            \begin{enumerate}
                \item Métodos sobre imágenes estáticas\\
                     Local Binary Patterns (LBP)\\
                    Este método consiste en realizar una codificación del píxel de interés con respecto a los valores de sus vecinos. En simples palabras se realiza un proceso de máscara en el cual se restan los píxeles del vecindario con el píxel central, si la resta es negativa o cero se asigna un cero en la posición del vecino, por el contrario si es positiva se asigna un uno. Luego de esto, se realiza una concatenación de los vecinos y se obtiene un código binario que es transformado a base 10. Dicho número es asignado como nuevo valor del píxel visitado.

Luego de realizar el proceso en todos los píxeles se procede a realizar un histograma de los valores el cual es definido como el descriptor de la imagen, siendo este utilizado una para posterior clasificación.

%Otros métodos que utilizan LBP antes de la obtención del descriptor dividen la imagen en $n$ secciones de igual o distintas áreas, y realizan el cálculo del histograma para cada una de estas secciones. Luego concatenan cada uno de los histogramas en un orden específico y se obtiene un macro descriptor más detallado de la imagen.

                \item Métodos sobre imágenes dinámicas\\
                    Volume Local Binary Patterns (VLBP)\\
                    Es un método basado en LBP para imágenes dinámicas, el cual se basa en tres variables: $L$ la cantidad de cuadros que entran en la creación del patrón, $P$ el tamaño del vecindario a seleccionar y $R$ el radio del cual se escogen los vecinos. 
El método VLBP consiste en realizar una codificación basada en LBP, ahora incluyendo la variable temporal $L$ que indica desde qué cuadro se comienza a realizar la resta del píxel seleccionado con el central.
Éste es un proceso muy costoso debido a que mientras mayor sea la cantidad de vecinos a seleccionar, mayor será la cantidad de dígitos binarios, lo cual implica un mayor espectro de números resultantes en la codificación.\\
                    Local Binary Patterns - Three Ortogonal Planes (LBP-TOP)\\
                    Es un método basado en LBP para imágenes dinámicas, consiste en crear tres planos ortogonales que se intersectan en el píxel de interés, siendo estos el plano $XY$ , $XT$ y $YT$ (cuadro actual , movimiento temporal en $X$ y en $Y$ respectivamente). 

Para poder obtener el patrón de la imagen se realiza el mismo proceso de resta con los píxeles del vecindario seleccionado, pero a diferencia de los métodos estáticos que solo se realizan en el plano $XY$, este también se realiza para los planos $XT$ e $YT$. Con esto se obtiene un histograma para cada plano, los cuales son concatenados y forman el macro descriptor de la imagen.

            \end{enumerate}
        \item Entrenamiento y Clasificación\\
        Luego de realizar la extracción de características se procede a realizar la clasificación de estas utilizando algoritmos de clasificación como Support Vector Machines (SVM) y/o  Hidden Markov Model (HMM).

        \end{enumerate}    
        
        \end{enumerate}
    Teniendo ya claro como se realiza el reconocimiento de expresiones faciales, también existen técnicas utilizadas en el reconocimiento de patrones que pueden servir para crear nuevos métodos denominados ''Híbridos''.
    \begin{enumerate}
    \item Bag of Features (BOF) o Bolsa de características.\\
Esta técnica consiste en crear un universo de todas las características que son extraídas de los casos de entrenamiento, con lo cual se puede tener una representación del espacio vectorial en el cual se esta trabajando. Luego se realizan técnicas de clustering para poder crear grupos de características similares. 
Uno de los grandes problemas de la utilización de esta técnica es la parametrización del número de clusters que se requieren, ya que el valor óptimo de este parámetro se puede encontrar a través de ensayo y error. 
    \item Optical Flow\\
Técnica que permite estimar el movimiento de las regiones de interés en las imágenes dinámicas, en general consiste en realizar un seguimiento de la región de interés durante todos los cuadros y obtener un vector que permita representar dicho flujo del movimiento.
    \end{enumerate}


\section{Descripción de la solución}
\subsection{Método solución}
%La idea de este proyecto es poder crear un nuevo método ``Híbrido'' que permita reconocer expresiones faciales en imágenes dinámicas (videos), utilizando las técnicas descritas en el estado del arte. Al ser un método de reconocimiento de patrones, éste se divide en dos partes. Primero el entrenamiento del método, y luego la fase de pruebas.

Para este nuevo método se introduce un nuevo micro descriptor basado en técnicas similares al optical flow, el cual nos permitirá realizar el seguimiento de las regiones de interés o ROI (por sus siglas en inglés) que denominaremos ``rayo de flujo'' o simplemente ``rayo''.

La cadena de procedimientos ordenados o “pipeline” de este método se divide en cuatro grandes etapas:
    \begin{enumerate}
    \item Extracción de Micro-Descriptores\\
    Éste proceso se centrará en encontrar las regiones de interés del rostro (ROI), luego para cada una de éstas se realizará un seguimiento con respecto al $i$ cuadro siguiente y se verá cual es el movimiento de éste con respecto al cuadro actual, esto permitirá realizar el cálculo de un rayo de flujo. Al realizar el cálculo de los rayos para cada cuadro se obtendrá un conjunto de rayos que representan la ROI de la imagen.
    \item Normalización de rayos y creación del bag of features (BoF)\\
Luego de la extracción del conjunto de rayos de cada una de las ROI se procede a realizar la normalización de los rayos de tal forma que cada conjunto pueda ser leído en el mismo espacio vectorial.

Luego de normalizar los rayos de cada vídeo, se procede a utilizar la técnica del BoF, dicha técnica tendrá como entrada cada uno de los rayos normalizados obtenidos en el paso anterior. Posteriormente crearemos un número $w$ de clusters que permiten asignar un valor a cada uno de los rayos generados de los vídeos. 

    \item Generación de macro descriptores\\
Para poder crear cada uno de estos macro descriptores se procede a realizar un histograma de la imagen resultante del BoF, siendo este el vector de características que será utilizado por el clasificador SVM tanto para entrenar como para realizar las pruebas del sistema.

En el caso de que en la etapa 2 no se logre encontrar una normalización eficiente de los rayos, en este paso se procederá a crear un descriptor por cada uno de los vídeos los cuales serán ingresados al clasificador HMM, el cual permite ingresar los descriptores de forma dinámica.
	
    \item Utilización y prueba de Clasificadores\\
Al ser un proyecto netamente dedicado a la investigación, se necesitará realizar pruebas con más de un clasificador para poder obtener una gran cantidad de resultados que puedan ser comparados con los métodos propuestos en el estado del arte hasta antes de comenzar la memoria. 

Los algoritmos contra los cuales se va a realizar la comparación son los mencionados en los antecedentes.

En esta etapa se procederá a entrenar los métodos propuestos anteriormente, y luego se realizaran las pruebas utilizando técnicas como $K$-fold cross-validation.
    \end{enumerate}
    
Cabe destacar que esta investigación es de carácter exploratorio, por lo cual, se probaran distintas soluciones a cada una de las etapas anteriormente descritas. Lo cual permitirá probar distintas permutaciones entre cada una de estas soluciones. Al ser un proyecto con un limite determinado de tiempo, hemos decidido que solo se probaran las combinaciones que den mejores resultados, de las cuales la que obtenga el mejor sera la seleccionada para el método final de la memoria. 

En el caso de no encontrar ninguna posible solución se procederá a describir cada una de las técnicas que se utilizaron, sus resultados y sus respectivas soluciones.  

\subsection{Objetivos} 
\begin{enumerate}
\item General\\

Crear un descriptor para el reconocimiento de expresiones faciales en vídeo utilizando micro patrones basados en el seguimiento del flujo de los movimientos del rostro.

\item Específicos
    \begin{enumerate}
    \item Lograr definir un método de elección de las regiones de interés del rostro.
    \item Lograr definir una codificación que permita la normalización de los rayos de flujo.
    \item Crear una métrica que permita la comparación entre rayos luego del proceso de normalización.
    \item Utilizar y probar distintos algoritmos de clasificación.
    \item Evaluar y comparar el método creado con el estado del arte hasta antes de comenzar la memoria.
    \end{enumerate}
\end{enumerate}

\section{Metodología de trabajo}
Al ser un proyecto de investigación, éste esta sujeto a variaciones debido a que se utiliza el método de ensayo y error. La metodología que se utilizara serán dos reuniones semanales para discutir temas relevantes y muestra de avances del proyecto.
	El proyecto se divide en las siguientes etapas: Diseño del pipeline, Implementación del pipeline, esta etapa implica implementar cada uno de los módulos descritos en la solución, Diseño de casos de estudio, Evaluacion y comparacion del metodo, Conclusiones.

\section{Cronograma de actividades, hitos y entregables}
\begin{tabular}{ll}

\hline\noalign{\smallskip}

Fecha & Actividad \\

\hline\noalign{\smallskip}

24/03/14 - 07/04/14 & Análisis del estado del arte.\\

07/04/14 - 28/04/14 & Creación del prototipo.\\

28/05/14 - 19/05/14 & Primera iteración del prototipo.\\

19/05/14 - 09/06/14 & Segunda iteración del prototipo, diseño de pruebas y obtención de resultados.\\

09/06/14 - 27/06/14 & Elaboración del informe.\\

27/06/14 - 18/07/14 & Correcciones y presentación.\\

11/08/14 - 01/09/14 & Tercera iteración del prototipo e implementación de algoritmos del estado del arte.\\

01/09/14 - 29/09/14 & Cuarta iteración del prototipo y comparación con algoritmos implementados.\\

29/09/14 - 27/10/14 & Quinta iteración del prototipo y recolección de resultados.\\

27/10/14 - 17/11/14 & Sexta iteración del prototipo, recolección de resultados y elaboración del informe.\\

17/11/14 - 28/11/14 & Elaboración y entrega del informe.\\

28/11/14 - 05/12/14 & Correcciones y presentación.\\
\hline


\end{tabular}


\section{Resultados esperados}
\begin{enumerate}
\item Poder crear un método de reconocimiento de expresiones faciales basado en una nueva forma de codificación de patrones, el cual pueda tener una efectividad igual y/o mejor que los métodos del estado del arte.
\item Lograr encontrar un método de normalización de los rayos de flujo que permita la mínima pérdida de la información.
\item Lograr crear una métrica de comparación entre los rayos de flujo que permita la mínima pérdida de la información.
\item Lograr un valor óptimo para el número de clusters necesarios en el BoF.

\end{enumerate}
\nocite{*}
\bibliographystyle{plain}
\bibliography{references}
\end{document}
