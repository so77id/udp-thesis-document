% Copyright 2004 by Till Tantau <tantau@users.sourceforge.net>.
%
% In principle, this file can be redistributed and/or modified under
% the terms of the GNU Public License, version 2.
%
% However, this file is supposed to be a template to be modified
% for your own needs. For this reason, if you use this file as a
% template and not specifically distribute it as part of a another
% package/program, I grant the extra permission to freely copy and
% modify this file as you see fit and even to delete this copyright
% notice. 

\documentclass{beamer}


\usepackage[utf8]{inputenc}
\usepackage[spanish]{babel}
\usepackage{lmodern}

\usepackage{graphicx}
\usepackage{multicol}
\usepackage{mathtools}
\usepackage{hyperref}
\usepackage{listings}
\usepackage{verbatim}
\usepackage{afterpage}
\usepackage{lscape}

\usepackage{multirow, array} % para las tablas

\usepackage{amsmath}

\usepackage{cite}



% Establecemos el tema a utilizar. 
% Debe existir el archivo udpthesisEIT.sty en su sistema TeX para poder utilizarlo.

%%%%%%%%%%%%%%%%%%%%%%%%%%%%%%%
\usepackage{xspace}
\makeatletter
\DeclareRobustCommand\onedot{\futurelet\@let@token\@onedot}
\def\@onedot{\ifx\@let@token.\else.\null\fi\xspace}

\def\eg{\emph{e.g}\onedot} \def\Eg{\emph{E.g}\onedot}
\def\ie{\emph{i.e}\onedot} \def\Ie{\emph{I.e}\onedot}
\def\cf{\emph{c.f}\onedot} \def\Cf{\emph{C.f}\onedot}
\def\etc{\emph{etc}\onedot} \def\vs{\emph{vs}\onedot}
\def\wrt{w.r.t\onedot} \def\dof{d.o.f\onedot}
\def\etal{\emph{et al}\onedot}
\def\adhoc{\emph{ad hoc}\xspace}
\makeatother

\usepackage{amsthm}
% declare my operator
\usepackage{scalerel}
\DeclareMathOperator*{\cat}{\scalerel*{\|}{\sum}}



% There are many different themes available for Beamer. A comprehensive
% list with examples is given here:
% http://deic.uab.es/~iblanes/beamer_gallery/index_by_theme.html
% You can uncomment the themes below if you would like to use a different
% one:
%\usetheme{AnnArbor}
%\usetheme{Antibes}
%\usetheme{Bergen}
%\usetheme{Berkeley}
%\usetheme{Berlin}
%\usetheme{Boadilla}
%\usetheme{boxes}
%\usetheme{CambridgeUS}
%\usetheme{Copenhagen}
%\usetheme{Darmstadt}
%\usetheme{default}
%\usetheme{Frankfurt}
%\usetheme{Goettingen}
%\usetheme{Hannover}
%\usetheme{Ilmenau}
%\usetheme{JuanLesPins}
%\usetheme{Luebeck}
\usetheme{Madrid}
%\usetheme{Malmoe}
%\usetheme{Marburg}
%\usetheme{Montpellier}
%\usetheme{PaloAlto}
%\usetheme{Pittsburgh}
%\usetheme{Rochester}
%\usetheme{Singapore}
%\usetheme{Szeged}
%\usetheme{Warsaw}

\setbeamertemplate{footline}[page number]{}


\title{Reconocimiento de expresiones faciales en imágenes dinámicas utilizando un descriptor basado en rayos de flujo}

% A subtitle is optional and this may be deleted
%\subtitle{Optional Subtitle}

\author{Miguel Antonio Rodríguez Santander}
% - Give the names in the same order as the appear in the paper.
% - Use the \inst{?} command only if the authors have different
%   affiliation.

\institute[Universidad Diego Portales] % (optional, but mostly needed)
{%
  Universidad Diego Portales\\
  Facultad de Ingeniería\\
  Escuela de Informática y Telecomunicaciones
  
  % - Use the \inst command only if there are several affiliations.
% - Keep it simple, no one is interested in your street address.
}
\date{Título I\\ 30 de Julio de 2013}
% - Either use conference name or its abbreviation.
% - Not really informative to the audience, more for people (including
%   yourself) who are reading the slides online

\subject{Thesis}
% This is only inserted into the PDF information catalog. Can be left
% out. 

% If you have a file called "university-logo-filename.xxx", where xxx
% is a graphic format that can be processed by latex or pdflatex,
% resp., then you can add a logo as follows:

% \pgfdeclareimage[height=0.5cm]{university-logo}{university-logo-filename}
% \logo{\pgfuseimage{university-logo}}

% Delete this, if you do not want the table of contents to pop up at
% the beginning of each subsection:

\AtBeginSection[]
{
  \begin{frame}<beamer>{Índice}
    \tableofcontents[currentsection,currentsubsection]
  \end{frame}
}

%\AtBeginSubsection[]
%{
%  \begin{frame}<beamer>{Índice}
%    \tableofcontents[currentsection,currentsubsection]
%  \end{frame}
%}

% Let's get started
\begin{document}

\begin{frame}
  \titlepage
\end{frame}

\begin{frame}{Índice}
  \tableofcontents
  % You might wish to add the option [pausesections]
\end{frame}

% Section and subsections will appear in the presentation overview
% and table of contents.


% You can reveal the parts of a slide one at a time
% with the \pause command:

\section{Motivación}

\begin{frame}
  \frametitle{Motivación}
  \begin{columns}[onlytextwidth]
    \begin{column}{0.5\textwidth}
      \centering
      \includegraphics[width=5cm]{imagenes/google_glass.jpg}
    \end{column}
    \begin{column}{0.5\textwidth}
        \begin{itemize}
            \item Aplicaciones a distintas áreas del conocimiento humano
            \item Utilidad en tecnologías actuales
            \item 55\% en la transmisión del mensaje
        \end{itemize}
    \end{column}
  \end{columns}
\end{frame}


\section{Problema}
    
    \begin{frame}{Problema}{Expresiones faciales universales}
		  \begin{columns}[onlytextwidth]
    		  	\begin{column}{0.7\textwidth}
      			\centering
      			\includegraphics[width=7cm]{imagenes/expresiones_faciales_universales.jpg}
    			\end{column}
    		    \begin{column}{0.3\textwidth}
        	      \begin{itemize}
            	    \item Alegría
            		\item Asco
            		\item Ira
            		\item Miedo
            		\item Sorpresa
            		\item Tristeza
        	   	  \end{itemize}
            \end{column}
          \end{columns}          
          
          
          
    \end{frame}
    

    \subsection{Soluciones}
        \begin{frame}{Métodos Geométricos}
               \begin{columns}[onlytextwidth]
    		  		\begin{column}{0.5\textwidth}
      				\begin{figure}[bt]
        					\centering
                			\includegraphics[width=4cm]{imagenes/geo1.jpg}
            			\end{figure}	
      			\end{column}
			    \begin{column}{0.5\textwidth}
      				\begin{figure}[bt]
			        		\centering
            			    \includegraphics[width=4cm]{imagenes/geo2.jpg}
            			\end{figure}	
      			\end{column}
			   \end{columns}      			
        \end{frame}
    
        \begin{frame}{Métodos de apariencia en imágenes}{Local Binary Patterns}
            \begin{figure}[bt]
        		\centering
                \includegraphics[width=9cm]{imagenes/lbp.pdf}
            \end{figure}	

            \begin{figure}[bt]
        		\centering
                \includegraphics[width=9cm]{imagenes/lbp_histogram.png}
            \end{figure}	  
        \end{frame}
        
        \begin{frame}{Métodos de apariencia en imágenes}{Local Direccional Number Patterns}
            \begin{figure}[bt]
        		\centering
                \includegraphics[width=10cm]{imagenes/ldn.jpg}
            \end{figure}
        \end{frame}
        
        
        \begin{frame}{Métodos de apariencia en videos}{Volume Local Binary Patterns}
            \begin{figure}[bt]
        		\centering
                \includegraphics[width=7cm]{imagenes/vlbp.pdf}
            \end{figure}
        \end{frame}
    
        
        \begin{frame}{Métodos de apariencia en videos}{Local Binary Patterns- Three Ortogonal Planes}
            \begin{figure}[bt]
        		\centering
                \includegraphics[width=10cm]{imagenes/lbptop.pdf}
            \end{figure}
        \end{frame}
    

\section{Solución propuesta}
	\subsection{Pipeline}    
    \begin{frame}{Solución propuesta}{Pipeline}
        \begin{figure}[bt]
    		\centering
            \includegraphics[width=12cm]{imagenes/pipeline.png}
        \end{figure}
    \end{frame}
	
    \subsection{Preprocesamiento de la base de datos}
        \begin{frame}{Preprocesamiento de la base de datos}
        
            \begin{columns}[onlytextwidth]
                \begin{column}{0.4\textwidth}
                    \begin{itemize}
                        \item Detección de rostros
                        \item Corrección de movimiento
                    \end{itemize}
                \end{column}
                \begin{column}{0.6\textwidth}
                    \begin{figure}[bt]
                		\centering
                        \includegraphics[width=7cm]{imagenes/Preprocesamiento.png}
                    \end{figure}
                \end{column}
            \end{columns}
        
        \end{frame}
    
    
    
    \subsection{Extracción de micro-descriptores}
        \begin{frame}{Extracción de micro-descriptores}{Vista general}
            \begin{figure}[bt]
        		\centering
                \includegraphics[width=12cm]{imagenes/Extractor_microdescriptores.png}
            \end{figure}
        \end{frame}
    
    
        \begin{frame}{Extracción de micro-descriptores}{Extracción de rayos}
			\begin{block}{Región de soporte}
            		Región cuadrada de tamaño $L$ centrada en el píxel $(x,y)$ en el tiempo $t$
            \end{block}    			
    			
    			\begin{figure}[bt]
        			\centering
                \includegraphics[width=9cm]{imagenes/Extraccion_de_rayos.png}
            \end{figure}        
	        
        \end{frame}
    
    
    		\begin{frame}{Extracción de micro-descriptores}{Extracción de rayos}
			\begin{block}{Ventana de búsqueda}
                  Región cuadrada de tamaño $2L+1$ centrada en el píxel $(x,y)$ en el tiempo $t+1$
            \end{block}    			
    			
    			\begin{figure}[bt]
        			\centering
                \includegraphics[width=9cm]{imagenes/Extraccion_de_rayos.png}
            \end{figure}        
	            
        \end{frame}
    
        \begin{frame}{Extracción de micro-descriptores}{Extracción de rayos}
			\begin{equation}\label{algoritmo:eq:mse}	
			    \text{MSE}(\text{RS}, \text{RS}') = \sum_{x=1}^{L} \sum_{y=1}^{L} (\text{RS}(x,y,t) - \text{RS}'(x',y', t+1))^2
		    \end{equation}   
		    
		    \begin{equation}
		        \text{RS}^* = \arg \min_{\text{RS}'}\{\text{MSE}(\text{RS},\text{RS}') | \text{RS}' \in \text{WS} \}
	        \end{equation}
	        
	        \begin{columns}[onlytextwidth]
                \begin{column}{0.5\textwidth}
					\begin{figure}[bt]
        					\centering
                			\includegraphics[width=4cm]{imagenes/MSE1.png}
            			\end{figure}      	        	
	        		\end{column}
                \begin{column}{0.5\textwidth}                
                		\begin{table}
                			
					\begin{tabular}{|c|c|}
					                \hline 
					                RS & Región de soporte \\ 
					                \hline 
					                WS & Ventana de búsqueda \\ 
					                \hline 
					                RS' & Subregión de soporte $\in$ WS \\ 
					                \hline 
					                $x$ & Coordenada x del píxel \\ 
					                \hline 
					                $y$ & Coordenada y del píxel \\ 
					                \hline 
					                $\text{RS}^*$ & Subregión óptima $\in$ WS \\
					                \hline 
					 \end{tabular} 
                		\end{table}
	        		\end{column}
	        \end{columns}	
        \end{frame}
        
        
		\begin{frame}{Extracción de micro-descriptores}{Extracción de rayos}
			\begin{block}{Rayo de soporte}
				\begin{equation}
					\rho_{(x,y)} = \{(\Delta x^{t}, \Delta y^{t})~| \forall t\}.
				\end{equation}		 
            \end{block}			
			
			\begin{columns}[onlytextwidth]
                \begin{column}{0.5\textwidth}
					\begin{figure}[bt]
        					\centering
                			\includegraphics[width=4cm]{imagenes/MSE2.png}
            			\end{figure}      	        	
	        		\end{column}
                \begin{column}{0.5\textwidth}
                		\begin{align}
						\Delta x^{t} &= x-x^*\\ 
						\Delta y^{t} &= y-y^*
					\end{align}
	
					\begin{table}
                			
					\begin{tabular}{|c|c|}
					                \hline 
					                $x^*$ & Coordenada x de $\text{RS}^*$ \\ 
					                \hline 
					                $y^*$ & Coordenada y de $\text{RS}^*$ \\ 
					                \hline 
					 \end{tabular} 
                		\end{table}					
					
	        		\end{column}
	        \end{columns}			
			

            
				            
            
		\end{frame}		        
        
        
        \begin{frame}{Extracción de micro-descriptores}{Extracción de rayos}
			\begin{block}{Rayo de flujo}
				\begin{equation}
					R(x,y)	 = \{\rho_{(x,y)}, \rho_{(x^*,y^*)}, \rho_{(x^{**},y^{**})}, ... \},
				\end{equation}
		    \end{block}			
			
			\begin{figure}[bt]
        			\centering
                \includegraphics[width=9cm]{imagenes/Extraccion_de_rayos.png}
            \end{figure}             
            
            
        \end{frame}
       

        \begin{frame}{Extracción de micro-descriptores}{Normalización de rayos}
            \begin{figure}[bt]
        		\centering
                \includegraphics[width=10cm]{imagenes/normalizacion_de_rayos.png}
            \end{figure}
        \end{frame}
    
    \subsection{Creación de macro-descriptores}
		\begin{frame}{Creación de macro-descriptores}{Vista general}
            \begin{figure}[bt]
        		\centering
                \includegraphics[width=8cm]{imagenes/Extractor_macrodescriptores_entrenamiento.png}
          		\caption{Creación de macro-descriptores entrenamiento.}
            \end{figure}
            
            \begin{figure}[bt]
        		\centering
                \includegraphics[width=8cm]{imagenes/Extractor_macrodescriptores_clasificacion.png}
          		\caption{Creación de macro-descriptores clasificador.}
            \end{figure}
        \end{frame}
        
        
        \begin{frame}{Creación de macro-descriptores}{Bag of Visual Words}
            \begin{equation}
  				\label{algoritmo:eq:dist}
				k^* = \arg \min_k \{\mathit{dist}(R(x,y),C_k)\},
			\end{equation}
            \begin{figure}[bt]
        		\centering
                \includegraphics[width=7cm]{imagenes/bow_solo.png}
            \end{figure}
        \end{frame}
        
        \begin{frame}{Creación de macro-descriptores}{Creación del descriptor}
			\begin{equation}
				D_i(k) = \sum_{(x,y)\in\text{ROI}_i} \delta (R(x,y),k) \quad \forall k
			\end{equation}
			\begin{equation}
				 \delta (R(x,y),k) =  \begin{cases}
				 1 & \mbox{si }R(x,y)~\in~C_k\\
     			0 & \text{otro caso}
     			\end{cases}
			\end{equation}
		
			\begin{equation}
				\mathbb{D} = \cat_{i = 1}^{I} D_i, \quad \forall i
			\end{equation}				
			
			\begin{columns}[onlytextwidth]
                \begin{column}{0.3\textwidth}
				    \begin{figure}[tb]
        					\centering
                			\includegraphics[width=3cm]{imagenes/rois.png}
            			\end{figure}
	        		\end{column}
            		\begin{column}{0.7\textwidth}
				    \begin{figure}[tb]
        					\centering
                			\includegraphics[width=7cm]{imagenes/macro-descriptor.png}
            			\end{figure}
            		\end{column}
            	\end{columns}
        \end{frame}
    \subsection{Entrenamiento y clasificación}
    


		%\begin{frame}{Entrenamiento y clasificación}{Support Vector Machines}
        %   \begin{figure}[bt]
        %	\centering
        %       \includegraphics[width=8cm]{imagenes/svm.jpg}
        % 		\caption{Utilización de Kernel para el cambio de espacio.}
        %   \end{figure}
        %\end{frame}
        
        \begin{frame}{Entrenamiento y clasificación}{Support Vector Machines}
            \begin{figure}[bt]
        		\centering
                \includegraphics[width=5cm]{imagenes/support_vector_machines.png}
            \end{figure}
        \end{frame}
        
        \begin{frame}{Entrenamiento y clasificación}{k-fold cross validation}
            \begin{figure}[bt]
        		\centering
                \includegraphics[width=7cm]{imagenes/K-fold.jpg}
            \end{figure}
            \begin{equation}
            		Precision = \frac{|Videos~bien~clasificados|}{|Total~de~videos|} 
            \end{equation}
   
        \end{frame}
        
        
	\subsection{Ventajas potenciales del método}
		\begin{frame}{Solución propuesta}{Ventajas potenciales del método}
			\begin{columns}[onlytextwidth]
   	 			\begin{column}{0.5\textwidth}			
					\begin{itemize}
						\item Problema de la variable temporal
						\item Comportamiento de los píxeles				
						\item Modelado de micro-patrones				
						\item Rayos tipo			
					\end{itemize}
				\end{column}
				\begin{column}{0.5\textwidth}			
					\begin{figure}[bt]
        					\centering
                			\includegraphics[width=5cm]{imagenes/fotos-de-rayos.jpg}
            			\end{figure}
				\end{column}

			\end{columns}
		\end{frame}	
			        
  \section{Conclusiones}
	\begin{frame}{Conclusiones}
		\begin{itemize}
			\item Demasiados datos ruidosos
			\item Baja precisión
			\item Alto número de rayos por video
			\item Utilizar solo regiones del rostro: nariz, boca, ceja, etc
			\item Normalización tiende a deformar los datos (pérdida de datos y adición de ruido)
			\item Rol de la variable K
			\item Uso de memoria
		\end{itemize}
	\end{frame}      

\section{Trabajos futuros}
        \begin{frame}{Trabajos futuros}{Experimentos}
			\begin{enumerate}
				\item Codificación
					\begin{enumerate}
						\item Sin codificación
						\item Codificación LBP
						\item Codificación LDN
					\end{enumerate}
				\item Variables
					\begin{enumerate}
						\item Tamaño región de soporte $L$
						\item Variable de normalización $N$
						\item Cantidad de \textit{Clusters}
					\end{enumerate}
				\item Métricas de distancia.
			\end{enumerate}					        
        \end{frame}
    



% All of the following is optional and typically not needed. 
\section*{Preguntas}
	\begin{frame}{Gracias por su atención}
		\begin{center}
			¿Preguntas?
		\end{center}
	\end{frame}

\section*{Anexo}
	\begin{frame}{Anexo I}{Definiciones}
		\begin{block}{Región de soporte}
                  Dado un píxel $(x,y)$ en un cuadro $t$, definimos una región de soporte RS, de tamaño $L \times L$ (donde $L$ es impar), como la subregión del cuadro $t$ que está centrada en el píxel $(x,y)$.
         \end{block}
         \begin{block}{Ventana de búsqueda}
                  Dado un píxel $(x,y)$ en un cuadro $t$, definimos la ventana de búsqueda WS, de  tamaño $(2L+1) \times (2L+1)$, como la subregión del cuadro $t+1$ que está centrada en el píxel $(x,y)$.
         \end{block}
    	\end{frame}
	\begin{frame}{Anexo I}{Definiciones}
		\begin{block}{Rayo de soporte}
                Dado un par ordenado $(x,y)$ en el tiempo $t$, el \textit{Rayo de soporte} para dicho píxel esta definido por el nuevo par ordenado $(\Delta x^{t}, \Delta y^{t})$, de tal forma que,
			\begin{equation}
				\rho_{(x,y)} = \{(\Delta x^{t}, \Delta y^{t})~| \forall t\}.
			\end{equation}		 
        \end{block}   
    \end{frame}
	\begin{frame}{Anexo I}{Definiciones}
	    \begin{block}{Rayo de flujo}
                Es el conjunto de \textit{Rayos de soporte} que representan el movimiento del píxel $(x,y)$ en cada uno de los cuadros $t$, este conjunto es definido como \textit{Rayo de flujo} o micro-descriptor, de tal forma que,
			\begin{equation}
				R(x,y)	 = \{\rho_{(x,y)}, \rho_{(x^*,y^*)}, \rho_{(x^{**},y^{**})}, ... \},
			\end{equation}
		donde ($x^{**}$,$y^{**}$) es el píxel al cual se desplazo ($x^{*}$,$y^{*}$) en el cuadro $t+2$.  
        \end{block}      
    \end{frame}	

			\begin{frame}{Anexo II}{Entrenamiento y clasificación}
			  \begin{columns}[onlytextwidth]
   	 			\begin{column}{0.5\textwidth}
		            \begin{figure}[bt]
    			    			\centering
            		    		\includegraphics[width=6cm]{imagenes/Entrenamiento.png}
          				\caption{Diagrama de entrenamiento.}
            			\end{figure}
 				\end{column}
    				\begin{column}{0.5\textwidth}
		            \begin{figure}[bt]
        					\centering
        		        		\includegraphics[width=6cm]{imagenes/Clasificacion.png}
        	  				\caption{Diagrama de clasificación.}
		            \end{figure}
        			\end{column}
 			  \end{columns}
        \end{frame}


\end{document}


