% Ejemplo del uso de latemplate para escribir tesis/memorias de la Universidad Diego Portales.
%
% Eviar bugs a: Adín Ramírez, adin.ramirez (at) mail.udp.cl

% Puede generar borradores si omite la opción "final" de la clase.
%\documentclass{udpthesis}

\PassOptionsToPackage{table}{xcolor}
\RequirePackage{xcolor} % [table] 
\documentclass[final]{udpthesis}


% Establecemos el sistema para uso del español
% Babel ya esta cargado dentro de updthesis
\usepackage[T1]{fontenc} % output
\usepackage[utf8]{inputenc}% input
\usepackage{lmodern}
\usepackage{dsfont}

\usepackage{afterpage}
\usepackage{lscape}
% Agregue aca otros packetes que le sean de utilidad
% Matemáticas
\usepackage{amsmath}

% Gráficos
\usepackage{graphicx}
% \usepackage{subfig}

% Código
% \usepackage{listings}

% Referencias
\usepackage{cite}

\usepackage{subcaption}

% Gráficos, tablas, etc.
\usepackage{pgfplots, pgfplotstable, booktabs, colortbl, siunitx, array}
\sisetup{table-auto-round}

% Establecemos el tema a utilizar. 
% Debe existir el archivo udpthesisEIT.sty en su sistema TeX para poder utilizarlo.
\udptheme{EIT}


%%%%%%%%%%%%%%%%%%%%%%%%%%%%%%%
\usepackage{xspace}
\makeatletter
\DeclareRobustCommand\onedot{\futurelet\@let@token\@onedot}
\def\@onedot{\ifx\@let@token.\else.\null\fi\xspace}

\def\eg{\emph{e.g}\onedot} \def\Eg{\emph{E.g}\onedot}
\def\ie{\emph{i.e}\onedot} \def\Ie{\emph{I.e}\onedot}
\def\cf{\emph{c.f}\onedot} \def\Cf{\emph{C.f}\onedot}
\def\etc{\emph{etc}\onedot} \def\vs{\emph{vs}\onedot}
\def\wrt{w.r.t\onedot} \def\dof{d.o.f\onedot}
\def\etal{\emph{et al}\onedot}
\def\adhoc{\emph{ad hoc}\xspace}
\makeatother

\usepackage{amsthm}
\newtheorem{definition}{Definición}[chapter]

% declare my operator
\usepackage{scalerel}
\DeclareMathOperator*{\cat}{\scalerel*{\|}{\sum}}


%%%%%%%%%%%%%%%%%%%%%%%%%%%%%%%
%\cat_{i = 1} {N}

\begin{document}
%% Inicio de la portada
\frontmatter
% Título del tema
\title{Reconocimiento de expresiones faciales en imágenes dinámicas utilizando un descriptor basado en rayos de flujo}

% El autor(es) de la tesis
\author{Miguel Antonio Rodríguez Santander}
\email{so77id@gmail.cl}% utilice un correo que revise después de graduado
% o una lista de autores
%\author{Juan Bar\\José Foo}

% Fecha a aparecer en la tesis
\date{Diciembre, 2014}

% Profesor guía
\professor{Adín Ramírez}
% Comité evaluador
% puede tener uno o dos evaluadoers (deje en blanco el parámetro que no utilizará)
\committee{ David Röthlisberger }{ Andrea Nieto }
% \committee{Nicola Tesla}{Isaac Newton}

% Dedicatoria
\dedicatory{\textit{``Violence is the last refuge of the incompetent.''} - Isaac Asimov.}
%
% Agradecimientos
\acknowledgment{Agradezco a mi familia por ser fuente de apoyo constante e incondicional en toda mi vida y más aún en mis duros años de estudio.
A mi profesor guía Adín Ramírez por la motivación de siempre estar educándome para ser un mejor profesional y por su paciencia.
A mis amigos por apoyarme incondicionalmente en mis aventuras y desventuras, especialmente a Felipe Troncoso, Rodrigo Fuenzalida y Camilo Contreras quienes ayudaron al desarrollo de esta tesis.}

% Generamos la portada
\makecover

% Indices y listas
\tableofcontents% tabla de contenido
\listoffigures%   índice de figuras
\listoftables%    índice de tablas
% puede agregar otras listas o índices acá de ser necesario

% Abstrac en inglés
\begin{abstract}
The facial expression recognition is a research area in constant growth, due to the large number of new technologies thahat have recently been developed. Current technological advances in this area allow us to create or enhance human-computer interfaces, which help to better quality of life for people. Applications in areas such as medicine, marketing, psychology, robotics, public safety, among others, generate a high demand for this kind of technologies.

In this thesis a new temporal descriptor for recognition of the six universal expressions (Joy, Disgust, Anger, Fear, Sadness and Surprise) is proposed, for which a new term called ray flux, which consists of simple words is introduced recoding based on movement of the important areas of human face through video pictures used. The creation of this new descriptor consists of four stages: preprocessing of the database, that is, applying face recognition and motion correction techniques; Removing micro-descriptors, at which stage the extraction process flow rays on all video frames is performed; Creating macro-descriptors, after removal of micro-descriptors proceeds to create the descriptors for each video general, using techniques such as Bag of Words and clustering; and finally a training and classification step in which the model using Support Vector Machines, and then use this model in the classification of new entries trains.
\end{abstract}

% Resumen
\begin{resumen}
El reconocimiento de expresiones faciales es un área de investigación en constante crecimiento, debido a la gran cantidad de nuevas tecnologías que se pueden desarrollar. Los actuales avances tecnológicos en esta área permiten crear o mejorar interfaces humano-computador, las cuales ayudan a la mejor calidad de vida de las personas. Aplicaciones en áreas como la medicina, el marketing, la psicología, la robótica, seguridad pública entre otras, tienen una alta demanda de tecnologías de este estilo.

En esta tesis se propone un nuevo descriptor espacio-temporal para el reconocimiento de las seis expresiones universales (Alegría, Asco, Ira, Miedo Sorpresa y Tristeza), para el cual se introduce un nuevo término llamado \textit{Rayo de flujo}, el cual consiste en una nueva codificación  basada en el movimiento de las áreas importantes del rostro humano a través de los cuadros del vídeo analizado. La creación de este nuevo descriptor consta de cuatro etapas:  \emph{Preprocesamiento de la base de datos}, la cual consiste en aplicar técnicas de reconocimiento de rostro y corrección de movimiento sobre los datos de entrada. \emph{Extracción de micro-descriptores}, etapa en la cual se realiza el proceso de extracción de \textit{rayos de flujo} sobre todos los cuadros del vídeo. \emph{Creación de macro-descriptores}, luego de la extracción de micro-descriptores se procede a crear los descriptores generales de cada vídeo, utilizando técnicas como \textit{Bag of Words} y \textit{Clustering}. Y por último \emph{Entrenamiento y clasificación}, etapa en la cual se entrena el modelo utilizando  las \emph{Support Vector Machines}, para luego utilizar este modelo en la clasificación de las nuevas entradas. En general los resultados obtenidos por nuestro método no superan el 48\% de \textit{Accuracy}, lo cual es un resultado pobre al ser comparado con métodos del estado del arte, los cuales obtienen valores superiores al 88\%. Por esto proponemos realizar mejoras en la etapa de extracción de \textit{rayos}, tales como solo utilizar la componente vertical de los movimientos o a su vez utilizar el angulo de movimiento para la creación de los \textit{rayos}, ademas también proponemos realizar una segmentación de las regiones importantes del rostro (ojos, nariz y boca), a las cuales se realiza todo el procesamiento del algoritmo.
\end{resumen}



% Inicio del contenido
\mainmatter

% Capítulos y secciones del documento
% Aca se incluyen los archivos con el texto de los capitulos
% Incluyo el archivo cap-intro.tex
\chapter[Introducción]{Introducción}
\label{ch:intro}

\section{Motivación}
\label{sec:motivacion}
Con los actuales avances tecnológicos, el reconocimiento de expresiones faciales (FER, Facial Expression Recognition) forma parte importante de una rama de la inteligencia artificial, específicamente el reconocimiento de patrones.  Las investigaciones sobre el FER permiten el desarrollo de futuras tecnologías ligadas al desarrollo de interfaces centradas en el humano. 

Existen diversas aplicaciones tecnológicas enfocadas al reconocimiento de expresiones faciales aplicadas en distintas áreas del conocimiento humano, tales como:  la medicina, que se utiliza para la detección de enfermedades y rehabilitación de trastornos mentales; la psicología, para reportes sobre estados de ánimo; el marketing, la posibilidad de una segmentación del cliente al cual se quiere llegar con la publicidad; la robótica, la interacción humano-robot; la seguridad, para combatir fraudes de pasaportes, soporte al orden público e identificación de personas desaparecidas; entre otras áreas.

Diversas empresas también han utilizado estos avances de reconocimiento facial como un componente que aporta información en las tecnologías que ofrecen. Un ejemplo de esto es  Google que utiliza el FER en su motor de búsqueda de imágenes y se proyecta utilizar este reconocimiento para aportar información a la entrada de video de los Google Glass~\cite{GoogleGlass}. Otro ejemplo es Facebook que usa la detección de rostros en  fotos para realizar el etiquetado rápido de usuarios.

Los estudios de Mehrabian~\cite{Mehrabian1968} sobre la comunicación oral indican que las expresiones faciales contribuyen con aproximadamente el 55\% de la transmisión del mensaje, lo cual es un valor mucho mayor que la parte verbal con 7\% y la parte vocal un 38\%.

El objetivo de este proyecto es buscar un nuevo descriptor espacio temporal para reconocer expresiones faciales. Para esto se propone un nuevo método utilizando imágenes dinámicas, el cual utilizaremos para reconocer las seis expresiones faciales universales: alegría, asco, ira, miedo, sorpresa y tristeza.

\section{Problema}
\label{sec:problema}

En la actualidad existen muchos métodos que permiten resolver el problema del reconocimiento de expresiones faciales, algunos utilizan el sistema de codificación de acciones faciales (FACS) creado por Ekman~y~Friesen~\cite{Ekman1978}, entre los cuales están las investigaciones de Pantic y Rothkrantz~\cite{Pantic2004}, Lien et al.~\cite{Lien1998}, Pantic y Patras~\cite{Pantic2006}, entre otros. Otras investigaciones utilizan métodos de reconocimiento basados en características geométricas o de apariencia, entre los cuales están Ramírez et al.~\cite{RamirezRivera2013}, utilizando Local Direccional Number Patterns; Lyons et al.~\cite{Lyons1998}, utilizando filtros de Gabor; Ahonen et al.~\cite{Ahonen2006}, utilizando Local Binary Patterns (LBP por sus siglas en inglés).

Las últimas investigaciones se centran en poder resolver el problema del reconocimiento de expresiones agregando la variable temporal $t$, la cual indica que ya no solo se quiere reconocer en imágenes sino que en secuencias de imágenes o videos, con lo cual además de necesitar crear algoritmos que permitan buscar la expresión a través del tiempo, también deben ser algoritmos eficientes, debido a la gran cantidad de cálculos que se deben realizar. Existen distintas investigaciones que intentan resolver el problema del reconocimiento en vídeos: Zhao y Pietikäinen~\cite{Zhao2006}, crean una nueva codificación basada en LBP introduciendo la variable del tiempo, esta codificación recibe el nombre de Volume Local Binary Patterns; Zhao y Pietikäinen~\cite{Zhao2006}, también crean otra codificación basada en LBP en la cual utilizan tres planos ortogonales, llamada LBP-TOP; Buenaposada~\etal~\cite{Buenaposada2008}, estiman la deformación de los componentes no rígidos del rostro; Yeasin~\etal~\cite{Yeasin2004}, crean una firma temporal de cada entrada, con la cual entrenan sus modelos; Xiang~\etal~\cite{Xiang2008}, utilizan la transformada de Fourier para realizar la extracción de características, y luego utilizan Fuzzy C-means para generar los modelos.

En este trabajo proponemos un nuevo método que nos permita poder resolver el problema del reconocimiento de expresiones faciales en videos. La idea general de nuestra solución es juntar métodos de distintas áreas de las ciencias de la computación y crear una nueva forma de representar las expresiones faciales en video. En general introducimos un nuevo micro-descriptor que llamamos \textit{rayo de flujo}, el cual junto con técnicas de aprendizaje maquina (machine learning, en inglés), recuperación de información (information retrieval, en inglés), visión por computador (computer vision, en inglés) y reconocimiento de patrones (pattern recognition, en inglés), nos permitirán crear un modelo que realice el reconocimiento de las seis expresiones faciales universales. 

 %ésta debe ser una solución robusta, puesto que es necesario poder lograr resultados comparables con los actuales métodos ya creados. Esta solución debe poder responder preguntas como ¿Es posible crear un modelo de clasificación a partir de las teorías propuestas en este trabajo?, de ser así, ¿Qué tan buena es la representación creada por el modelo?, ¿Cómo se compara con los algoritmos que ya resuelven este problema?, ¿Qué tan eficiente es con respecto al tiempo de ejecución es el algoritmo?, ¿Es posible implementar esta idea para reconocimiento en tiempo real?, \etc.



\section{Solución propuesta}
\label{sec:solucion}

Para este nuevo método se introduce un nuevo micro-descriptor, el cual nos permite realizar el seguimiento de las regiones de interés (ROI, regions of interest) que denominaremos \textit{rayo de flujo} o simplemente \textit{rayo}.


\subsection{Pipeline}

La cadena de procedimientos ordenados o \textit{pipeline} de este método se divide en cuatro grandes etapas: preprocesamiento de la base de datos, extracción de micro-descriptores, creación de macro-descriptores, y entrenamiento y clasificación, este procedimiento puede ser visto en la Figura~\ref{intro:fig:pipeline}.
	\begin{figure}[b]
		\centering
		\includegraphics[width=1\textwidth]{Figuras/Diagramas/pipeline.png}
		\caption{Pipeline del algoritmo propuesto.}
		\label{intro:fig:pipeline}
	\end{figure}		
	
\textbf{Preprocesamiento de la base de datos.}
Antes de poder realizar la extracción de características de los videos, es necesario realizar un preprocesamiento sobre éstos, esto debido a que los videos pueden tener ciertos elementos que estropean la extracción de características. Para evitar agregar ruido al video se utilizan dos técnicas que permiten obtener solo el rostro de la persona a representar su expresión y a su vez corregir los movimientos que éstos puedan tener. Esta etapa será profundizada en la Sección~\ref{sec:proc_bdd}.


\textbf{Extracción de micro-descriptores.}
Este proceso consiste en la extracción de los rayos de flujo de cada uno de los videos preprocesados, para esto, primero se seleccionan las regiones de interés de la cara. Luego, para cada uno de los pixeles de la región, se procede a calcular el movimiento de este píxel con respecto al cuadro siguiente, y así sucesivamente con cada uno de los cuadros del vídeo. Luego de calcular el movimiento para cada píxel en cada cuadro, se obtiene el conjunto de \textit{rayos} o micro-descriptores que definen el video. Esta etapa será profundizada en la Sección~\ref{sec:micro_descriptores}.


\textbf{Creación de macro-descriptores.}
Luego de extraer cada uno de los micro-descriptores, se procede a crear utilizar la técnica de la bolsa de palabras visuales (\textit{Bag of Visual Words} en inglés), este método es utilizado para poder agrupar los \textit{rayos} en $k$ grupos, los cuales definen el espacio de los macro-descriptores. Cada \textit{rayo} extraído de los vídeos pertenece únicamente a un solo grupo, por lo tanto es posible para cada vídeo obtener un histograma de la frecuencia de aparición de cada uno de estos grupos en sus \textit{rayos}. Este histograma obtenido es el macro-descriptor que se utiliza para poder entrenar y clasificar. Esta etapa será profundizada en la Sección~\ref{sec:macro-descriptores}.


\textbf{Entrenamiento y clasificación.}
Luego de obtener los macro-des\-crip\-tores para cada uno de los vídeos, se procede a la etapa de entrenamiento y posterior clasificación. Para poder entrenar y clasificar con los macro-descriptores obtenidos en la etapa anterior, se utilizan las Maquinas de Vectores de Soporte (Support Vector Machines por sus siglas en ingles). Esta técnica permite dividir el espacio vectorial representado por los descriptores de entrenamiento, para luego ser probado por los descriptores que se obtienen en la clasificación o prueba del modelo creado. Para poder probar la efectividad de la clasificación se utiliza la técnica de validación cruzada, la cual permite realizar distintas pruebas de entrenamiento y clasificación, para obtener cual es el porcentaje de efectividad del método. Esta etapa será profundizada en la Sección~\ref{sec:clasificacion}.

\subsection{Ventajas potenciales del método}
\label{intro:ventajas}
En esta investigación se introduce un nuevo micro-descriptor llamado \textit{Rayo de flujo}, en simples palabras, 
un \textit{rayo de flujo} es el conjunto de variaciones que tiene un píxel determinado a lo largo del vídeo, esto será explicado en mayor profundidad en la Sección~\ref{sec:micro_descriptores}.

Éste, al ser un enfoque nuevo no visto en otras investigaciones, las distintas ventajas potenciales serán evaluadas a lo largo de la investigación, en general nos centraremos en demostrar o desmentir las siguientes incógnitas:

\begin{itemize}
	\item ¿Permitirán los \textit{rayos} realizar un modelado espacio-temporal?
	\item Al ser un modelado espacio-temporal de los píxeles, los \textit{rayos}, ¿permitirán ver cual es el comportamiento de los pixeles a través del tiempo?
	\item ¿Permitirán los \textit{rayos} modelar micro-patrones en los movimientos del rostro que no pueden ser vistos a simple vista por los humanos?, de ser cierto, ¿estos micro-patrones aportarán mayor información al modelo de clasificación?.
	\item ¿Existe la posibilidad de que cada una de las expresiones faciales universales pueda tener asociado un conjunto de \textit{rayos} que la definan?. 
\end{itemize}


\section{Objetivos}
\label{subsec:objetivos}
A continuación se describen los objetivos de este trabajo:

\subsection{Objetivo general}
\label{subsubsec:objgeneral}
Crear un descriptor espacio-temporal basado en el seguimiento del movimiento de los pixeles para el reconocimiento de expresiones faciales en video.

%Crear un descriptor para el reconocimiento de expresiones faciales en vídeo utilizando micro patrones basados en el seguimiento del flujo de los movimientos del rostro.
%\item Crear o encontrar una métrica que permita la comparación entre rayos luego del proceso de normalización.
%\item Evaluar y comparar el método creado con el estado del arte hasta antes de comenzar la memoria. 
%\item Definir un método de elección de las regiones de interés del rostro.
%\item Definir un método de elección de las regiones de interés del rostro.

\subsection{Objetivos específicos}
\label{subsubsec:objgeneral}
	\begin{enumerate}
		\item Modelar el comportamiento de los pixeles a través del tiempo.
		\item Definir una codificación que permita la normalización de los rayos de flujo.
		\item Estudiar la distribución de los movimientos de los pixeles sobre el rostro.
	\end{enumerate}
% Incluyo el archivo cap-tema.tex
\chapter[Marco teórico]{Marco teórico}
\label{ch:estado_del_arte}

\section{Reconocimiento de patrones}
\label{sec:rec_patrones}
	No existe una definición universalmente aceptada para el Reconocimiento de Patrones (RP o PR, pattern recognition en inglés). Esta puede recibir distintas definiciones dependiendo de la disciplina sobre la cual se quiere aplicar. Algunos autores como Duda~y~Hart~\cite{Duda1973} definen el reconocimiento de patrones, junto con reconocimiento de máquina como un campo preocupado de las regularidades del ruido en entornos complejos. Theodoridis~y~Koutroumbas~\cite{Theodoridis2008} lo define como una disciplina científica que apunta a la clasificación de objetos dentro de un conjunto de categorías o clases, y también como una parte integral del sistema de inteligencia de la máquina construida para la toma de decisiones. González y Thomason~\cite{Gonzalez1978} define el PR como la clasificación de la entrada de datos a través de la extracción de importantes características de un conjunto de estos con ruido. 

\begin{figure}[b]
  \centering
   \includegraphics[width=1\textwidth]{Figuras/Diagramas/estado_del_arte/Reconocimiento_de_patrones.png}
  \caption{Arquitectura básica de un sistema de reconocimiento de patrones.}
  \label{art:fig:arquitectura}
\end{figure}


En este trabajo se define el reconocimiento de patrones como un área de la Inteligencia Artificial (IA o AI, artificial intelligence en inglés) enfocada en la extracción de características de un conjunto de imágenes o vídeos, las cuales definen los patrones que serán utilizados para la creación de un modelo que permita clasificar nuevas entradas. Podemos describir este procedimiento en tres procesos como se puede ver en la Figura~\ref{art:fig:arquitectura}: adquisición de datos, extracción de características y toma de decisiones. 

	\textbf{Adquisición de datos.} La adquisición de datos es el procedimiento encargado de sensar la información del mundo real, para esto se utilizan distintos sensores dependiendo del área al cual se quiera aplicar el PR. Estos sensores tienen distintas respuestas dependiendo de su calidad o las variables del medio ambiente como la luminosidad, ruido ambiental, \etc. Para este trabajo nos enfocaremos en los sensores de imágenes instalados en las cámaras que capturan los vídeos que luego serán clasificados por nuestro sistema.
	
	\textbf{Extracción de características.} La extracción de características es el primer proceso en el cual interviene la maquina, tiene como objetivo transformar las señales adquiridas por la etapa anterior a una representación vectorial capas de ser interpretada por los clasificadores para la toma de decisiones. Este procedimiento es el que mas cambia a la hora de construir sistemas de reconocimiento de patrones, esto debido a que existe una gran variación de la implementación a realizar dependido del problema que se requiera atacar. En general esta etapa consiste en realizar un estudio de las características que diferencian cada uno de los elementos que queremos reconocer, luego de esto, se busca alguna forma de poder cuantificar estas características computacionalmente, con tal de lograr una representación vectorial de cada entrada llamada descriptor. Para esta etapa existen distintos métodos que se pueden utilizar para resolver el problema propuesto, estos serán mayormente estudiados en la Sección~\ref{sec:type_fe}.
	
	\textbf{Toma de decisiones.} La toma de decisiones o clasificador es la ultima etapa del PR, esta se encarga de realizar la toma de decisiones utilizando las características extraídas en la etapa anterior. Para esto podemos dividir este bloque en dos sub-procesos, entrenamiento y clasificación. El entrenamiento consiste en tomar una gran cantidad de datos ya clasificados anteriormente de forma correcta por algún experto en el tema, y utilizar estos datos para poder crear un modelo el cual divida el espacio vectorial representado por las características de cada uno de los datos de entrenamiento. La clasificación consiste en utilizar el modelo creado en la etapa de entrenamiento para poder clasificar nuevas entradas no tomadas en cuenta en el entrenamiento. Para la etapa de creación del modelo se pueden utilizar distintas técnicas que permiten dividir el espacio en distintas regiones dependiendo de las etiquetas o clases a las cuales se requiera clasificar, tales como: Redes bayesianas, técnicas de clustering, redes neuronales, Modelos ocultos de Markov (HMM, hidden Markov models en inglés). Para el desarrollo de nuestro método utilizaremos las maquinas de soporte vectorial~\cite{Cortes1995,Hearst1998} (SVM, Support Vector Machines en inglés), las cuales permiten realizar una clasificación de muchas clases en un mismo espacio vectorial al mismo tiempo. 


	\subsection{Support Vector Machines}
	\label{sec:svm}
	Las Máquinas de Vectores de Soportes (SVM por sus siglas en inglés) son herramientas fundamentales en sistemas de aprendizaje automático, se basan en  implementar reglas de decisión complejas, por medio de un kernel que permite mapear los puntos de entrenamiento a un espacio de mayor dimensión, esta técnica es llamada \textit{kernel trick}, y consiste en realizar una transformación del espacio vectorial a analizar, donde se pueda encontrar una recta o hiperplano que permita dividir las muestras en dos grupos, este \textit{kernel trick}, puede ser visualizado en la Figura~\ref{art:fig:kernel_trick}, en donde vemos que por medio de un kernel radial, se logra encontrar una recta que separe las muestras en dos grupos. 
	
	Las máquinas son utilizadas para la clasificación y análisis de regresión permitiendo generar un modelo de predicción dado un conjunto de datos de entrada, y posteriormente utilizar este modelo para clasificar nuevos datos.
	
	\begin{figure}[tb]
		\centering
		\includegraphics[width=.5\textwidth]{Figuras/Diagramas/estado_del_arte/kernel_trick.jpg}
		\caption{Gráfica demostrativa del cambio en el espacio vectorial al utilizar un kernel.}
		\label{art:fig:kernel_trick}
	\end{figure}
	
	\begin{figure}[t]
		\centering
		\includegraphics[width=.5\textwidth]{Figuras/Diagramas/estado_del_arte/support_vectors.jpg}
		\caption{Gráfica demostrativa de la elección de los vectores de soporte.}
		\label{art:fig:support_vectors}
	\end{figure}
		
	La idea general de SVM es solucionar un problema de clasificación de dos clases mediante una barrera de decisión, para encontrar esta barrera se realiza la elección de vectores de soporte, los cuales serán los encargados de ayudar a solucionar el problema de optimización, esta elección puede ser vista en la Figura~\ref{art:fig:support_vectors}, donde las muestras encerradas en circulo representa a los vectores de soporte, los cuales ayudan a encontrar la barrera de decisión o hiperplano que separa las muestras en dos grupos, de tal forma que todos los vectores de soporte tienen la misma distancia a dicha barrera. Estos métodos evolucionaron con el tiempo y se utilizan para poder solucionar el problema de clasificación de múltiples clases.

	Existen distintos tipos de Kernels que permiten encontrar hiperplanos mas eficientes dependiendo de la representación de los datos de entrenamiento, algunos de los mas conocidos son:

	\textbf{Lineal}: Kernel lineal, no realiza ninguna transformación vectorial, es utilizado cuando las muestras son linealmente separables,
	\begin{equation}
	\kappa(x_i,x_j) = x_{i}^{T}x_j;
	\label{k:lineal}
	\end{equation}
	
	\textbf{RBF}: Kernel radial, es la mejor opción la mayoría de las veces, permite realizar una transformación a un espacio curvo, en el cual una recta o hiperplano de este espacio curvo es representada por circunferencia en el plano cartesiano, en este kernel se tiene una variable de suavizado llamada $\gamma$,
	\begin{equation}
	\kappa(x_i,x_j) = e^{-\gamma||x_i - x_j||^2 }, \gamma > 0;
	\label{k:RBF}
	\end{equation}
	
	\textbf{Poly}: Kernel polinomial, mapea en función del grado del polinomio, en general realiza una transformación a un espacio vectorial, en el cual una recta o hiperplano es representada por un polinomio de grado $degree$ en el plano cartesiano, la variable $coef0$ representa un corrimiento y $\gamma$ el suavizado,
	\begin{equation}
	\kappa(x_i,x_j) = (\gamma x_{i}^{T}x_j + coef0)^{degree}, \gamma > 0;
	\label{k:Poly}
	\end{equation}

	\textbf{Histogram Intersection}: Utilizado para la clasificación de histogramas, es generalmente utilizado cuando las muestras representan histogramas.
	\begin{equation}
	\kappa(x_i,x_j) = \min(x_i,x_j).
	\label{k:Inter}
	\end{equation}


Para efectos de este trabajo, SVM es utilizado para poder generar el modelo final que permita clasificar las seis expresiones faciales canónicas, utilizando los descriptores obtenidos luego de la extracción de características.

\section{Reconocimiento de expresiones faciales}
\label{sec:fer}
Las expresiones faciales constituyen una guía básica en la interacción social. Por ello, las alteraciones en su expresión o reconocimiento suponen una importante limitación para la comunicación. Ya que las expresiones del rostro son la variable que más se observa para obtener información de las emociones de nuestros interlocutores; si bien es cierto que tenemos un elevado control sobre nuestra expresividad facial, diversos estudios han demostrado que, cuando una persona está utilizando una expresión facial no acorde con su verdadero estado de ánimo, en su cara aparecen durante breves momentos señales de la expresión verdadera que a menudo pasan desapercibidas por las demás personas, estas se conocen como expresiones faciales universales.


\subsection{Expresiones faciales universales}
\label{sec:type_fe}

Las expresiones faciales del rostro humano tienen una infinidad de posibles variaciones que permiten la comunicación, regulación y adecuación de las emociones en el contexto social.

Darwin afirma en uno de  sus estudios publicado a fines del siglo XIX, que nosotros no podemos entender las expresiones emocionales humanas sin entender las expresiones emocionales de los animales, para esto, él argumenta que nuestras expresiones son determinadas en gran parte por nuestra evolución~\cite{Darwin1956,Darwin1998}. En la actualidad Paul Ekman, psicólogo pionero en el estudio de las emociones y su relación con la expresión facial, propone que las expresiones faciales están compuestas por micro-expresiones muy breves, que duran sólo una fracción de segundo. Estas se producen cuando una persona ya sea deliberadamente o inconscientemente esconde un sentimiento~\cite{Ekman1981}.

Ekman y Friesen definen que existen seis expresiones universales~\cite{Ekman2003}: 

\begin{enumerate}
	\item Alegría (Happiness): se produce mediante la contracción del músculo que va del pómulo al labio superior y del orbicular que rodea al ojo. También se elevan las mejillas. 
	\item Asco (Disgust): Existe una ligera contracción del músculo que frunce la nariz y estrecha los ojos. El gesto de la nariz arrugada es simultáneo al de la elevación del labio superior. 
	\item Ira (Anger): Se produce con una mirada fija, cejas juntas y hacia abajo, y existe una tendencia a apretar los dientes. 
	\item Miedo (Fear): Se observa los párpados superiores elevados al máximo e inferiores tensos. Las cejas levantadas se acercan y los labios se alargan hacia atrás. 
	\item Sorpresa (Surprise): Los párpados superiores suben, pero los inferiores no están tensos, la mandíbula suele caer. 
	\item Tristeza (Sadness): Se manifiesta cuando los párpados superiores caen y las cejas se angulan hacia arriba. El entrecejo se arruga y los labios se estiran de forma horizontal.
\end{enumerate}


\subsection{Métodos de reconocimiento de expresiones faciales}
Los Métodos de reconocimiento de expresiones faciales pueden ser divididos en dos grupos: métodos geométricos y métodos de apariencia, los geométricos consisten en extraer características de la geometría del rostro, por el contrario, los métodos de apariencia se encargan de extraer características de las texturas de la imagen y encontrar patrones ocultos. Para mas información acerca de los métodos de reconocimiento de expresiones faciales  ver otros estudios como los realizados por Pantic~y~Rothkrantz~\cite{Pantic2000} o Zeng~\etal~\cite{Zeng2009}. A su vez cada uno de estos grupos tiene métodos que ayudan a reconocer las expresiones en imágenes o en vídeos. Para esta investigación nos centramos en estudiar solo los métodos basados en apariencia.

\subsubsection{Métodos basados en características geométricas}
\label{sec:met_geo}
Los métodos geométricos se basan en detectar formas y regiones de la cara que se esta procesando, así como puntos de características faciales (\eg ojos, comisuras de la boca, \etc). 
Luego de encontrar las regiones importantes del rostro que se está analizando se procede a realizar la clasificación. Los métodos geométricos al ser aplicados sobre imágenes dinámicas o vídeos, realizan el procedimiento de encontrar cada una de estas regiones importantes del rostro que aportan la mayor cantidad de información, y luego se realiza el seguimiento a lo largo de la secuencia de cuadros del vídeo. Una de las grandes ventajas de estos métodos, es que al ser enfocado en encontrar una cantidad determinada de puntos, son algoritmos muy rápidos y de fácil aplicación a sistemas de reconocimiento en tiempo real. A su vez no son tan robustos como los basados en apariencia.


\subsubsection{Métodos basados en características de apariencia}
\label{sec:met_apa}
Se basan en detectar los movimientos de las texturas en la imagen, en el caso de imágenes con rostros humanos, cambios o deformaciones en la piel (\eg arrugas, protuberancias, surcos, \etc). En si son métodos que no se basan en encontrar ciertos puntos específicos del rostro, sino que encuentran regiones importantes de las texturas a analizar. Existen distintas aplicaciones de estos métodos, tanto sobre imágenes estáticas como para vídeo o imágenes dinámicas.  

	\subsubsection{Métodos basados en características sobre imágenes}
	\label{sec:met_imagen}
		Son métodos enfocados en encontrar ciertas regularidades existentes entre las texturas de la imagen a analizar, ejemplos:

		\subsubsection{Local Binary Patterns}
		\label{sec:lbp}
		Método introducido por Wang y He~\cite{Wang1990}, consiste en realizar una codificación del píxel de interés con respecto a los valores de sus vecinos. En simples palabras se realiza un proceso de máscara en el cual se restan los píxeles del vecindario con el píxel central, si la resta es negativa o cero se asigna un cero en la posición del vecino, por el contrario si es positiva se asigna un uno. Luego de esto, se realiza una concatenación de los vecinos y se obtiene un código binario que es transformado a base 10. Dicho número es asignado como nuevo valor del píxel visitado~\cite{Ojala1994,Ojala2002,Ahonen2004,Shan2009}. Este procedimiento puede ser visto en la Figura~\ref{art:fig:lbp}.
		
\begin{figure}[tb]
  \centering
   \includegraphics[width=1\textwidth]{Figuras/lbp.pdf}
  \caption{Proceso de codificación de LBP~\cite{Ahonen2006}.}
  \label{art:fig:lbp}
\end{figure}

Luego de realizar el proceso en todos los pixeles de la imagen se procede a realizar un histograma de los nuevos valores, este histograma es el descriptor de la imagen, el cual es utilizado para entrenar los modelos o simplemente ser clasificado.

Otros métodos que utilizan LBP antes de la obtención del descriptor dividen la imagen en $n$ secciones de igual o distintas áreas, y realizan el cálculo del histograma para cada una de estas secciones. Luego concatenan cada uno de los histogramas en un orden específico y se obtiene un macro descriptor más detallado de la imagen~\cite{Ahonen2006}.

		\subsubsection{Local Directional Number Pattern}
		\label{sec:ldn}
		Método introducido por Ramírez \etal~\cite{RamirezRivera2013}, consiste en realizar una codificación de los píxeles utilizando toda la vecindad. Utiliza el operador de Kirsch~\cite{Kirsch1971}, el cual permite encontrar las gradientes o bordes de la imagen las ocho direcciones. El método realiza un proceso de máscara para cada uno de los detectores de bordes de Kirsch, luego para cada píxel a codificar se obtiene el índice de la mascara que obtuvo la mayor intensidad al ser aplicada, a su vez también se obtiene el índice de la mascara que obtuvo el menor valor. Luego estos índices son transformados a una base binaria de tres dígitos y concatenados obteniendo una cadena binaria de seis dígitos, esta es nuevamente transformada a decimal y este valor decimal (entre $0$ y $63$) es el nuevo valor que representa el píxel, este proceso puede ser visto en la Figura~\ref{art:fig:ldn}, donde en la parte izquierda de la imagen podemos ver el proceso de aplicación de las ocho mascaras de Kirsch a la imagen, luego para cada pixel a codificar se escogen el mayor y menor valor resultante en las mascaras, los cuales con concatenados y el resultado es la codificación del píxel o nuevo valor resultante.
		
\begin{figure}[tb]
  \centering
   \includegraphics[width=1\textwidth]{Figuras/ldn.jpg}
  \caption{Proceso de codificación de LDN~\cite{RamirezRivera2013}.}
  \label{art:fig:ldn}
\end{figure}

		Luego de codificar toda la imagen, se realiza un histograma de la imagen, este es utilizado como descriptor para la posterior clasificación. También la imagen puede ser segmentada por una grilla, en la cual se calcula un histograma para cada una de las grillas, y luego se concatenan en un orden específico.

	\subsubsection{Métodos basados en características sobre imágenes dinámicas o videos}	
	\label{sec:met_videos}
	Para estos métodos se introduce la variable temporal $t$, la cual es la encargada de regir el movimiento de los píxeles a lo largo del tiempo, al igual que los métodos sobre imágenes estáticas, estos métodos resuelven el problema planteado de forma similar, pero adaptando las técnicas a la nueva variable.

		\subsubsection{Volume Local Binary Patterns.}
		\label{sec:vlbp}
		Método introducido por Zhao y Pietikäinen~\cite{Zhao2006,Zhao2007a,Zhao2007}, basado en LBP para imágenes dinámicas, el cual se basa en tres variables: $L$ la cantidad de cuadros que entran en la creación del patrón, $P$ el tamaño del vecindario a seleccionar y $R$ el radio del cual se escogen los vecinos. 
El método VLBP consiste en realizar una codificación basada en LBP, ahora incluyendo la variable temporal $L$ que indica desde qué cuadro se comienza a realizar la resta del píxel seleccionado con el central.
Éste es un proceso muy costoso debido a que mientras mayor sea la cantidad de vecinos a seleccionar, mayor será la cantidad de dígitos binarios, lo cual implica un mayor espectro de números resultantes en la codificación. Este proceso puede ser visto en la Figura~\ref{art:fig:vlbp}.

\begin{figure}[tb]
  \centering
   \includegraphics[width=0.7\textwidth]{Figuras/vlbp.pdf}
  \caption{Proceso de codificación de VLBP utilizando $L$~=~1, $P$~=~4 y $R$~=~1~\cite{Zhao2007}.}
  \label{art:fig:vlbp}
\end{figure}


		\subsubsection{Local Binary Patterns-Three Ortogonal Planes}
		\label{sec:lbp-top}
		 Es un método basado en LBP para imágenes dinámicas, consiste en crear tres planos ortogonales que se intersectan en el píxel de interés, siendo estos el plano $XY$, $XT$, y $YT$ (cuadro actual, movimiento temporal en $X$ y en $Y$ respectivamente). 

Para poder obtener el patrón de la imagen se realiza el mismo proceso de resta con los píxeles del vecindario seleccionado, pero a diferencia de los métodos estáticos que solo se realizan en el plano $XY$, este también se realiza para los planos $XT$ e $YT$.\@ Con esto se obtiene un histograma para cada plano, los cuales son concatenados y forman el macro descriptor de la imagen~\cite{Zhao2007}. Este proceso puede ser visto en la Figura~\ref{art:fig:lbptop}.


\begin{figure}[tb]
  \centering
   \includegraphics[width=1\textwidth]{Figuras/lbptop.pdf}
  \caption{Proceso de codificación de LBP-TOP~\cite{Zhao2007}.}
  \label{art:fig:lbptop}
\end{figure}


\section{Reconocimiento y corrección de rostros}
\label{sec:rec_rostros}

	\subsection{Framework Viola-Jones}
	\label{sec:viola-jones}
	Introducido por Viola y Jones~\cite{Jones2003}, es un método que utiliza clasificación en cascada, y un entrenador de clasificadores débiles basado en AdaBoost~\cite{Freund1995}. Este algoritmo consiste en subdividir la imagen a analizar en pequeñas sub-imágenes, las cuales son entregadas a una serie de clasificadores débiles (etapas), cada uno con un conjunto de características visuales. El método consiste en que cada sub-imagen es evaluada por los clasificadores débiles de la cascada, si en una es encontrado un rostro, esta sub-imagen es entregada al siguiente clasificador, el cual revisa de forma mas rigurosa la imagen. Si la imagen es aceptada por todos los clasificadores, esta es clasificada como un rostro.
	
	\subsection{Corrección de movimiento rígido}
	\label{sec:rigid}
	Los métodos de corrección de movimiento rígido consisten en utilizar técnicas de rotación y traslación de regiones de la imagen, esto para poder ajustar el movimiento de todos los cuadros del video sobre una región especifica. Este tipo de corrección no alteran el tamaño de la imagen, sólo su ubicación y el ángulo su rotación.


\section{Bag of Visual Words}
\label{sec:bag_of_words}
\textit{Bag of visual words}, es una técnica utilizada para la clasificación de documentos, consiste en la creación de un descriptor de cada documento a partir de la frecuencia de aparición de cada una de las palabras que este en la bolsa. Esta bolsa es construida como un diccionario que contiene todas las palabras que aparecen en los documentos a clasificar.

Recientemente esta técnica también es utilizada para la clasificación de imágenes y es conocida como Bag of Visual Words. Para poder utilizar la misma teoría, las palabras son reemplazadas por características extraídas de las imágenes. El problema es que la cantidad de características es muy alta y con una muy baja probabilidad de ser iguales entre ellas, por lo cual es necesario agrupar en conjuntos de características, las cuales son utilizados para construir los descriptores de las imágenes. Estos conjuntos de características son representados por un vector o centroide, el cual es el representante del grupo, luego de esto se procede a construir la bolsa de palabras agrupando todos estos centros como el vocabulario visual de las imágenes. Para poder encontrar los centroides de cada uno de los grupos es necesario utilizar técnicas de agrupamiento o \textit{clustering} como $k$-means. Implementaciones de esta técnica en clasificación de imágenes pueden ser vistas en distintas investigaciones~\cite{Csurka2004,Dollar2005,Sivic2009}.

	\subsection{$K$-means}
	\label{sec:k-means}
	$K$-means es un método introducido por Lloyd~\cite{Lloyd1982} utilizado mayormente en minería de datos, tiene como objetivo principal realizar una partición sobre un conjunto de datos en $k$ grupos, donde cada dato pertenece a solo un grupo.
	El algoritmo consiste en encontrar los centroides o \textit{clusters} de los $k$ grupos que se requiere crear, esto se realiza con un proceso iterativo, el cual dada la elección inicial de los centroides, calcula la distancia a cada uno de los datos, de tal forma que cada dato es introducido en un grupo. Luego recalcula el centroide de cada grupo como la media de cada uno de los datos que pertenecen a ese \textit{cluster}. Esto se realiza hasta cumplir con los requerimientos de parada, los cuales pueden ser: la inexistencia de nuevas asignaciones de grupo, pequeños cambios en los centroides o cambios mínimos en la suma del error cuadrático.
	Uno de los grandes problemas de este algoritmo, es la elección de los centroides iniciales, esto debido a que cada instancia puede encontrar distintos máximos locales, por lo cual esto se puede solucionar de forma que se corren distintas instancias con los mismos datos, y luego se calcula un promedio para cada uno de los $k$ centros.
	Otro de los grandes problemas de este método es la elección de la métrica de distancia a utilizar, ya que dependiendo de el tipo de los vectores a utilizar existen distintas formas de calcular la distancia, lo cual cambia los resultados de los resultados.
			
	%\subsubsection{Métricas de distancia}
	%\label{sec:matricas_de_distancia}
	%Las métricas de distancia se utilizan para saber que tan cerca o lejos esta un vector de otro, por lo cual esto permite poder tener la distancia entre dos vectores pertenecientes al mismo espacio vectorial. 
	%Sean dos vectores $p$-dimensionales $x$,$y$ de un subespacio de $\mathds{R}^p$. Entonces, se pueden definir las siguientes medidas de distancia entre $x$ e $y$~\cite{Pereira2010}:
	
%\paragraph{Distancia Euclidiana}\label{deuclidiana}

%\begin{equation}
%d(x,y) = \Big\{\sum_{i} (x_i-y_i)^2\Big\}^{1/2}
%\end{equation}


%\paragraph{Distancia Euclidiana %cuadrada}\label{deuclidianacuad}

%\begin{equation}
%d(x,y) = \sum_{i} (x_i-y_i)^2
%\end{equation}


%\paragraph{Distancia de Manhattan}\label{dmanhattan}

%\begin{equation}
%d(x,y) = \sum_{i} \vert x_i-y_i\vert
%\end{equation}

%\paragraph{Distancia de Minkowski}\label{dminkowski}

%\begin{equation}
%d(x,y) =\Big\{ \sum_{i} \vert x_i-y_i\vert^{1/l}\Big\}^l
%\end{equation}


%\paragraph{Distancia de potencia}\label{dpotencia}

%\begin{equation}
%d(x,y) = \Big\{\sum_{i} (x_i-y_i)^p\Big\}^{1/r}
%\end{equation}

%\paragraph{Distancia del coseno}\label{dcoseno}

%\begin{equation}
%d(x,y) = \frac{\sum_{i} (x_iy_i)}{\sqrt{\sum_{i} %x_i^2}\sqrt{\sum_{i} y_i^2}}
%\end{equation}


%\paragraph{Correlaci\'on de Pearson}\label{dpearson}

%\begin{equation}
%d(x,y) = \frac{\sum_{i} %(x_i-\bar{x})(y_i-\bar{y})}{\sqrt{\sum_{i} %(x_i-\bar{x})^2}\sqrt{\sum_{i} (y_i-\bar{y})^2}}
%\end{equation}


\chapter[Algoritmo]{Algoritmo}
\label{ch:algoritmo}
\section{Pipeline}
\label{sec:pipeline}
Para poder solucionar el problema propuesto en la Sección~\ref{sec:problema}, se propone un algoritmo compuesto por cuatro módulos. \textbf{Preprocesamiento de la base de datos:} modulo encargado de quitar todo el ruido de los vídeos de entrenamiento, \textbf{Extracción de micro-descriptores:} encargado de dividir el vídeo en distintas regiones de interés, para crear un micro-descriptor basado en el movimiento interno de estas, \textbf{Creación de macro-descriptores:} toma los micro-descriptores y crea un nuevo macro-descriptor por vídeo utilizando distintas técnicas y por ultimo \textbf{Entrenamiento y clasificación:} encargado de crear un modelo de clasificación con los descriptores de entrenamiento, para luego utilizar este modelo para la posterior clasificación de nuevos vídeos. Este proceso puede ser visto en la Figura~\ref{algoritmo:fig:pipeline}. 

	\begin{figure}[bt]
		\centering
    		\includegraphics[width=1\textwidth]{Figuras/Diagramas/pipeline.png}
  		\caption{Pipeline del algoritmo propuesto.}
  		\label{algoritmo:fig:pipeline}
	\end{figure}	


\section{Preprocesamiento de la base de datos}
\label{sec:proc_bdd}
	En esta etapa se procede a realizar la limpieza de todos los datos que entran al algoritmo. Dado que para el conjunto de datos de entrenamiento estamos utilizando una base de datos de vídeos (hacer referencias y hablar un poco de la BDD ya que se va hablar mas en los experimentos). Cada uno de estos vídeos de entrada tienen elementos que no aportan mayor información al método, sino que también pueden aportar ruido, de tal manera se procede a eliminar todos elementos. Para este procedimiento se divide en dos etapas: \textbf{Detección de rostros:} Utiliza el algoritmo de Viola-Jones para encontrar el rostro del primer cuadro, y segundo \textbf{Corrección de movimiento:} utiliza el cuadro encontrado del rostro para corregir el movimiento de todos los cuadros del vídeo. Este procedimiento puede ser visualizado en la Figura~\ref{algoritmo:fig:preprocesamiento}. 	
	
	\begin{figure}[tb]
		\centering
    		\includegraphics[width=1\textwidth]{Figuras/Diagramas/Preprocesamiento.png}
  		\caption{Diagrama del preprocesamiento de los datos.}
  		\label{algoritmo:fig:preprocesamiento}
	\end{figure}	

	
	\subsection{Detección de rostros}
	\label{algoritmo:det_rostro}
	Para cada uno de los vídeo que se requieran utilizar en el algoritmo es necesario quitar todos los elementos que no aporten información. Para realizar esta tarea primero es necesario utilizar un algoritmo de detección de rostros, que nos permita obtener los vértices del rectángulo que encierra la cara en el primer cuadro. Este rectángulo sera utilizado para la fase de corrección de movimiento como el modelo para centrar todos los cuadros del vídeo.	
		
	\subsection{Corrección de movimiento}
	\label{algoritmo:cor_movimiento}
	Luego de realizar la detección de rostros para cada uno de los cuadros del vídeo, se procede a realizar la corrección del movimiento rígida, la cual consiste en realizar operaciones de translación y rotación sobre cada uno de los cuadros del vídeo, utilizando la imagen del rostro detectado en el primer cuadro como modelo para poder ajustar los cuadros siguientes.


\section{Extracción de micro-descriptores}
\label{sec:micro-descriptores}
		El proceso de extracción de micro-descriptores es el núcleo del algoritmo. Este proceso es el encargado de la extracción de características directas del vídeo o imagen dinámica. Para esto se introduce un nuevo término que es denominado ``Rayo de flujo'' o simplemente ``Rayo''. Éste define cual es el movimiento de las regiones de interés (ROI sus siglas en inglés) en los cuadros del vídeo. Esta etapa se puede dividir en tres grandes bloques, primero \textbf{Codificación:} etapa en la cual el vídeo puede ser codificado utilizando técnicas como LBP o LDN. Esta etapa es opcional y sera mayormente profundizada en el Capitulo~\ref{ch:exp_result}, segundo \textbf{Extracción de rayos:} consiste en determinar las regiones de interés a las cuales se extraerán los ``Rayos de flujo'', y su posterior extracción. Y por ultimo \textbf{Normalización:} Esta etapa se encarga de llevar todos los ``Rayos de flujo'' al mismo espacio vectorial. Un diagrama de la Extracción de micro-descriptores puede ser visto en la Figura~\ref{algoritmo:fig:micro_descriptores}.

	\begin{figure}[tb]
		\centering
    		\includegraphics[width=1\textwidth,angle=90]{Figuras/Diagramas/Extractor_microdescriptores.png}
  		\caption{Diagrama general del proceso de extracción de micro-des\-crip\-to\-res.}
  		\label{algoritmo:fig:micro_descriptores}
	\end{figure}	


	\subsection{Elección de regiones de interés}
	\label{algoritmo:elecc_roi}
	Para cada uno de los vídeos que se quiere procesar, después de realizar el preprocesamiento definido en la Sección~\ref{sec:proc_bdd}, se seleccionan las regiones de interés a las cuales se les extrae los ``Rayos''. Las áreas son seleccionadas dependiendo la cantidad de información que entregan. 
	
	Definimos un vídeo como 
	\begin{equation}\label{algoritmo:eq:video}		
		V = \{\text{ROI}_i | 1 \leq i \le I\}, 
	\end{equation}
	donde $I$ es el número de regiones de interés en el vídeo.
	
	Definimos un ROI, $\text{ROI}_i$ como el conjunto de voxels (un voxel es una tripleta o un píxel en 3D), tal que
	\begin{equation}\label{algoritmo:eq:roi}
		\text{ROI}_{i}^{t} = \{(x,y) | (x,y,t) \in \mathds{R}(i)\},
	\end{equation}
	Donde $\mathds{R}(i)$, es una función que devuelve la $i$-esima región en que dividimos la imagen.

	\subsection{Extracción de rayos}
	\label{algoritmo:ext_rayos}
	
	Para esta sección se introducen dos términos, la región de soporte y ventana de búsqueda.
	La región de soporte es utilizada para calcular el movimiento del píxel central con respecto al cuadro siguiente. A su vez la ventana de búsqueda tiene como objetivo generar un espacio supuesto en el cual se puede haber desplazado la región de soporte en el siguiente cuadro.
	
	\begin{definition}[Región de soporte]	
	{Región de soporte.} Dado un píxel (x,y) en un cuadro t, existe una subregión cuadrada de tamaño $L$ (donde $L$ es impar) que esta centrada en dicho píxel, y además es subconjunto de $\text{ROI}_i^t$ tal que:
		\begin{equation}
			\text{RS}(x,y,t) := \{\text{RS} \subseteq \text{ROI}_i^t | (x,y,t) \in \text{ROI}_i^t \} 
		\end{equation}
	
	
%	es una sección de la ROI la cual es de tamaño $T$, donde $T$ es impar y esta centrada en el píxel $(x,y)$ en el cuadro $t$ del vídeo. Esta región es utilizada para calcular el movimiento de dicho píxel con respecto al cuadro $t+1$.
	\end{definition}

	\begin{definition}[Ventana de búsqueda]
	{Ventana de búsqueda.} Dado un píxel (x,y) en un cuadro t+1, existe una subregion cuadrada de tamaño $2L+1$ que esta centrada en dicho píxel, y ademas es subconjunto de $\text{ROI}_{i}^{t+1}$ tal que:
	\begin{equation}
		\text{WS}(x,y,t) := \{\text{WS} \subseteq \text{ROI}_{i}^{t+1} | (x,y,t+1) \in \text{ROI}_{i}^{t+1} \} 
	\end{equation}
	
	%al igual que la región de soporte es una sección del ROI de tamaño $2T+1$ y centrada en el píxel $(x,y)$, pero a diferencia de la anterior, esta se extrae del cuadro $t+1$. El objetivo de esta ventana es generar un espacio supuesto, donde se puede haber desplazo la región de soporte.
	\end{definition}
		
	La extracción de rayos consiste en que para cada píxel de la región de interés, se obtenga una región de soporte en cada cuadro y esta sea buscada dentro de la ventana de búsqueda del cuadro siguiente. Esta búsqueda se realiza calculando el error Cuadrático Medio o MSE (por sus siglas en ingles), \begin{equation}\label{algoritmo:eq:mse}	
			MSE(\mathit{RS},\mathit{RS'}) = \sum_{x=1}^{L} \sum_{y=1}^{L} (\mathit{RS}(x,y) - \mathit{RS'}(x,y))^2,
		\end{equation} 
	donde RS es la region de soporte del píxel (x,y) en el cuadro t, al que esta extrayendo el rayo, y RS' es una subregion de WS de tamaño $L$.
		
	Notemos que esta operacion se realiza para cada uno de los desplazamientos que se puedan realizar de la región de soporte sobre la de búsqueda. Luego de calcular el MSE de cada píxel dentro de la región de búsqueda se procede a calcular el minimo MSE de tal forma que la region RS' con el minimo MSE sera llamada RS*, tal que:
	\begin{equation}
		RS^* = \arg \min{RS'}\{\text{MSE}(\text{RS},\text{RS'}) | \text{RS'} \in \text{WS}\},
	\end{equation}		
	el cual será el cuadro donde la ventana de soporte se desplazó en el cuadro $t+1$, con esto se obtiene el píxel $(x^*,y^*)$ que corresponde al centro de la ventana obtenida en el proceso anterior.
	
	Para calcular el desplazamiento en el cuadro $t$ de $(x,y)$ a $(x^*,y^*)$, se procede a realizar la resta de los ejes. De tal forma que:
	\begin{align}
		\Delta x^{t} &= x-x^*,\\ 
		\Delta y^{t} &= y-y^*.
	\end{align}
		Donde $ \Delta x^*$ y $ \Delta y^*$ son componentes del ``Rayo de soporte'', y definimos el $j$-esimo ``Rayo de soporte'' como:
	\begin{equation}
		\rho_j = \{(\Delta x_j^{t}, \Delta y_j^{t})~| \forall t\},
	\end{equation}		
	donde j es el índice que equivale a enumerar los píxeles de la imagen de la forma:
	\begin{equation}
		j = Col*(y-1) + x,
	\end{equation}
	donde $Col$ es el numero columnas de píxeles que contiene la imagen.
	
	Los ``Rayos de soporte'' son la estructura básica de los ``Rayos de flujo'', dado que al juntar todos los desplazamientos del píxel $(x,y)$ a lo largo de los cuadros del vídeo, se obtiene un conjunto de variaciones de movimientos de estos. Este conjunto es definido como el ``Rayo'' o micro-descriptor.
	\begin{equation}
		R(x,y)	 = \{\rho_1(x,y), \rho_2(x^*,y^*), \rho_3(x^{**},y^{**}), ... \}
	\end{equation}
		Donde ($x^{**}$,$y^{**}$) es el píxel al cual se desplazo ($x^{*}$,$y^{*}$) en el cuadro $t$+2. El desplazamiento de los ``Rayos'' puede ser visto en la Figura~\ref{algoritmo:fig:normalizacion}.
		
		%%HABLAR DE ventana de soporte, ventana de busqueda MSE
		
	\subsection{Normalización de rayos}
	\label{algoritmo:normalizacion}
	Luego de obtener el total de micro-descriptores es necesario poder llevar todos los rayos al mismo espacio vectorial, esto debido a que el tamaño de cada conjunto depende de la cantidad de cuadros $T$ del vídeo. Para este proceso se introduce una variable muy importante para el algoritmo llamada $N$, esta sera la encargada de comandar el proceso de normalización, de tal forma que se obtiene una razón de normalización
	\begin{equation}
		Razon = \frac{T}{N},
	\end{equation}
	que permite relacionar la cantidad de ``Rayos de soporte'' $\rho_j$ del ``Rayo'' $R(x,y)$ original formaran parte de los $\rho_j '$ de $R(x,y)'$' normalizado. Este proceso puede ser visualizado en la Figura~\ref{algoritmo:fig:normalizacion}.
	
	\begin{figure}[bt]
		\centering
    		\includegraphics[width=1\textwidth]{Figuras/Diagramas/normalizacion_de_rayos.png}
  		\caption{Representación y Normalización de rayos.}
  		\label{algoritmo:fig:normalizacion}
	\end{figure}	

	
\newpage	
\section{Creación de macro-descriptores}
\label{sec:macro-descriptores}
El proceso de creación de macro-descriptores es el proceso final de la extracción de características de los vídeos. Luego de obtener un extenso conjunto de ``Rayos'' para cada uno de los vídeos, es necesario poder crear grupos de ``Rayos'', los cuales puedan representar de mejor manera el espacio ya normalizado. Para esto se utilizan técnicas de ``clustering'' o agrupamiento. El proceso de Creacion de maro-descriptores puede ser visto en las Figuras~\ref{algoritmo:fig:macro_descriptores:entrenamiento} y~\ref{algoritmo:fig:macro_descriptores:clasificacion}

	\begin{figure}[bt]
		\centering
    		\includegraphics[width=1\textwidth]{Figuras/Diagramas/Extractor_macrodescriptores_entrenamiento.png}
  		\caption{Proceso de creación de macro-descriptores en la fase de entrenamiento.}
  		\label{algoritmo:fig:macro_descriptores:entrenamiento}
	\end{figure}	
	
	
	\begin{figure}[bt]
		\centering
    		\includegraphics[width=1\textwidth]{Figuras/Diagramas/Extractor_macrodescriptores_clasificacion.png}
  		\caption{Proceso de creación de macro-descriptores en la fase de clasificación.}
  		\label{algoritmo:fig:macro_descriptores:clasificacion}
	\end{figure}	

	\subsection{Bag of words}
	\label{algoritmo:bow}
		Esta es una técnica dependiente de la cantidad de palabras $K$ o ``words'' que se introduzcan a la bolsa. Esta técnica permite armar $K$ grupos representados por un centroide o cluster llamado $C_k$ como se puede ver en la Figura~\ref{algoritmo:fig:bow}. Estos clusters son los puntos de referencia de cada uno de sus grupos. Para saber a cual de estos $k$ centroides pertenece un ``Rayo'' $R(x,y)$, es necesario calcular el
		\begin{equation}
  			\label{algoritmo:eq:dist}
			k^* = \arg \min\{dist(R(x,y),C_k)\},
		\end{equation}
		donde la función de distancia $dist()$ a utilizar depende de la instancia del algoritmo.

	\begin{figure}[tb]
		\centering
    		\includegraphics[width=1\textwidth]{Figuras/Diagramas/bow_solo.png}
  		\caption{Construcción del Bag of Words.}
  		\label{algoritmo:fig:bow}
	\end{figure}	

	\subsection{Creación de macro-descriptores}
	\label{algoritmo:crea_macro-descriptores}
	Luego de tener etiquetado cada uno de los ``Rayos'' $R(x,y)$ con su centroide respectivo, se procede a crear el macro-descriptor de cada uno de los vídeos.	 Esto se realiza creando un histograma de tamaño $K$ para cada uno de las regiones de interés ${ROI}_{i}^{t}$ del vídeo, de tal forma que en este histograma se tenga presente la frecuencia de cada uno de los clusters encontrados en la creación del ``Bag of Words''.
	\begin{equation}
  			\label{algoritmo:eq:hist}
  			D_i(k) = \sum_{(x,y)}^{} \delta (R(x,y),k), \forall k,
	\end{equation}
	
	\begin{equation}
		\label{algoritmo:eq:fun_hist}
		 \delta (R(x,y),k) = \left \{ \begin{matrix} 1 & \mbox{si }R(x,y)~\in~C_k
\\ 0 & otro~caso\end{matrix}\right. 
	\end{equation}

%		 \left \{ \begin{matrix} 1 & \mbox{si }R(x,y)\mbox{ \in C_k} \\ 0 & \mbox{}\mbox{de otra forma}\end{matrix}\right. 		

	
	Luego de obtener dichos descriptores por región, se procede a concatenar cada uno de los histogramas pertenecientes al mismo vídeo de la siguiente forma:
	\begin{equation}
		\mathds{D} = \cat_{i = 1}^{I} D_i, \forall i,
	\end{equation}	   
   este proceso puede ser visto en la Figura~\ref{algoritmo:fig:macro-descriptores}. El resultado de esta concatenación es el macro-descriptor del vídeo seleccionado.
	\begin{figure}[bt]
		\centering
  		\label{algoritmo:fig:macro-descriptores}
    		\includegraphics[width=1\textwidth]{Figuras/Diagramas/macro-descriptor.png}
  		\caption{Construcción del macro-descriptor.}
	\end{figure}	

	En el proceso clasificación, no se recalculan los centroides para un nuevo vídeo, ya que, se guarda el modelo que contiene el resultado del Bag of Words, y se procede a calcular~(\ref{algoritmo:eq:dist}), para cada uno de los ``Rayos'' $R(x,y)$ obtenidos en el proceso de extracción de micro-descriptores. Luego de esto al igual que en el proceso de entrenamiento se crea el macro-descriptor con el mismo método.
	
	
\section{Entrenamiento y clasificación}
\label{sec:clasificacion}
Ultima etapa del algoritmo, se dedicada a la segmentación del espacio vectorial formado por los macro-descriptores. Para este proceso se utilizara las Maquinas de Vectores de Soporte explicadas en la Sección~\ref{sec:rec_patrones}. Estas máquinas, se utilizan para entrenar el modelo que luego es el que permite clasificar las nuevas entradas de vídeo. 
Para poder probar la efectividad del clasificador se utilizaran técnicas de validación cruzada o $k$-fold cross-validation (Por sus siglas en inglés). Esta técnica consiste en dividir el conjunto de datos de datos en dos grupos, uno de entrenamiento y uno de prueba, $k$ veces. Se mostraran los resultados de esta técnica en la Capitulo~\ref{ch:exp_result}.


	\subsection{Entrenamiento}
	\label{algoritmo:entrenamiento}
		Luego de que el conjunto de datos de entrenamiento pasa por todas las etapas anteriores del algoritmo, se tiene para cada uno de los vídeos un macro-descriptor compuesto por la concatenación de los histogramas de de cada una de las regiones de interés y una etiqueta que indica a que clase pertenecen. Este conjunto de descriptores y las etiquetas son utilizados por las Maquinas de Vectores de Soporte para la creación de un modelo de clasificación, el permite poder etiquetar nuevas entradas, este proceso puede ser visto en la Figura~\ref{algoritmo:fig:entrenamiento}.
		
	\begin{figure}[bt]
		\centering
  		\label{algoritmo:fig:entrenamiento}
    		\includegraphics[width=0.7\textwidth]{Figuras/Diagramas/Entrenamiento.png}
  		\caption{Entrenamiento del clasificador.}
	\end{figure}	
		
		
	\subsection{Clasificación}
	\label{algoritmo:clasificacion}
		Ya teniendo el modelo entrenado resultante de la etapa de clasificación, este es utilizado para la clasificación de nuevas entradas sin etiquetar, esto con la misión de poder asignar una etiqueta al nuevo descriptor, este proceso puede ser visto en la Figura~\ref{algoritmo:fig:clasificacion}. En esta etapa se puede medir cual es la precision del algoritmo, esta medición sera mayormente abordada en el Capitulo~\ref{ch:exp_result}.
	
	\begin{figure}[bt]
		\centering
  		\label{algoritmo:fig:clasificacion}
    		\includegraphics[width=0.7\textwidth]{Figuras/Diagramas/Clasificacion.png}
  		\caption{Clasificador utilizando el modelo creado para clasificar nuevas entradas.}
	\end{figure}	
		
	
	
	
	
	

\chapter[Experimentos y resultados]{Experimentos y resultados}
\label{ch:exp_result}

\section{Base de datos MMI}
\label{exp:bdd}
Para poder realizar los experimentos con el algoritmo propuesto en el Capítulo~\ref{ch:algoritmo}, se utilizo una base de datos preparada para el reconocimiento facial. Creada por Pantic \etal~\cite{Pantic2005}, MMI es una base de datos multiuso que en esta instancia se utilizo para el reconocimiento de expresiones faciales. Contiene mas de 1000 ejemplos clasificados tanto en imágenes como vídeos; 19 sujetos de prueba; Las edades de los sujetos de prueba varían entre los 19 y los 62 años;  contiene tanto hombres como mujeres, además de tres razas étnicas distintas; Y por ultimo tanto las imágenes como los vídeos están grabados de forma frontal y lateral con respecto al rostro del sujeto.
Cada expresión facial esta etiquetada con una clase distintas: Ira(1), Asco(2), Miedo(3), Alegría(4), Tristeza(5) y Sorpresa(6).

\section{Experimentos}

Para poder probar la efectividad y buen modelado del algoritmo propuesto en el Capítulo~\ref{ch:algoritmo}, se preparo una pila de experimentos, los cuales permitieron realizar una revisión del modelado y precisión de método.
Se prepararon pruebas para cada uno de los pasos de los algoritmos, las cuales nos permitieron elegir los mejores valores para cada una de las variables.

Para la Extracción de micro-descriptores propuesta en la Sección~\ref{algoritmo:ext_rayos}, se realizaron pruebas que permitieron ver el modelado de los \textit{rayos de flujo} con respecto al movimiento de los pixeles a lo largo de los vídeos, y la elección del tamaño de la Región de soporte $R$ y la Ventana de búsqueda $W$.

En la Normalización de micro-descriptores, propuesto en la Sección~\ref{algoritmo:normalizacion}, se realizaron pruebas con distintos tamaños de $N$ (Variable utilizada para describir el nuevo tamaño de los \textit{rayos de flujo}) y se realizaron comparaciones con respecto a la Asertividad (Accuracy en ingles) de cada valor.

Para la creación de macro-descriptores, propuesta en la Sección~\ref{sec:macro-descriptores}, se prepararon dos experimentos distintos, primero se realizo un estudio del agrupamiento de los \textit{rayos} en el rostro para cada uno de los vídeos con distintos valores $K$, el cual indica la cantidad de grupos de \textit{rayos de flujo} existen; el segundo experimento que se realizo fue calcular la Asertividad (Accuracy en ingles) para distintos valores de la variable $K$.

Por ultimo en la etapa de entrenamiento y posterior clasificación, explicada en la Sección~\ref{sec:clasificacion}, se realizaron pruebas con los distintos \textit{Kernel} y sus respectivas variables, los cuales son recibidos por SVM para la generación del modelo.


\subsection{}


% % % % % % % % % % % % % % % % % % % % % % % % % % % % % % % % % % % % % % % % % % % % % % % % % % % %
% % % % % % % % % % % % % % % % % % % % B O R R A R % % % % % % % % % % % % % % % % % % % % % % % % % %
% % % % % % % % % % % % % % % % % % % % % % % % % % % % % % % % % % % % % % % % % % % % % % % % % % % % 
\subsection{Modelado del movimiento de los pixeles}
\label{exp:rayos}
Para poder probar la existencia de un real modelado del movimiento de los pixeles por parte de los \textit{rayos de flujo}, se creo un programa que recibe como entrada un video y el conjunto de pixeles a analizar su movimiento durante el video. Para esto se utilizo el algoritmo de creación de micro-descriptores analizado de la sección~\ref{sec:micro-_descriptores}. Luego de obtener los \textit{rayos}, se procedio a crear una imagen en los planos $XT$ e $YT$, cada uno de los cuales muestra el desplazamiento de un pixel a los largo del tiempo sobre un eje especifico $x$ e $y$ respectivamente.  

mientras mas grande la diferencia entre la RS y WS mayor es el error a la hora de moverse, por lo cual los rayos tienden a perderse.
con ventanas RS de tamaño 3 funciona bien con 5 y 7 de WS
con ventanas de rs tamaño 5 funciona bien con 7 y 9
con ventanas de rs tamaño 7 funciona bien con 11 y 13

Hablar de que con LBP se pierde los rayos

Mostrar graficos de movimiento en XT e YT
HABLAR DEL COMPORTAMIENTO DE LOS EJES, que el eje x modela, y el eje Y tiende a perderse.
Hablar de experimentos solo en el eje Y


\subsection{Obtencion del Kernel optimo para SVM}

Hablar sobre lo que se realizo para obtener los mejores valores.

Valores obtenidos para LINEAL y mostrar cuadro y grafico de la variable C v/s Accuracy
Valores obtenidos para RBF y mostrar cuadro y grafico para variable Gamma, C y Accuracy
valores obtenidos para Sigmoid y mostrar cuado de grafico para sus variables
valores obtenidos para Poly y mostrar cuajdro de graficos para sus variables

Hablar de Kernels para Histogramas y hablar de valores obtenidos con el kernel de comparacion de histograma


\subsection{Pruebas sobre variables del algoritmo}
\label{exp:var}
Como se explica en el Capítulo~\ref{ch:algoritmo}, el algoritmo cuenta con tres variables importantes, las cuales son la clave para la eficacia y precisión de este. Estas variables son el tamaño de la \textit{región de soporte} o variable $L$, el valor de la normalización de los \textit{rayos de soporte} o variable $N$ y por ultimo el tamaño del \textit{Bag of Visual Words} o variable $K$.

\subsubsection{Pruebas del tamaño de la región de soporte}
Como se explica en la Sección~\ref{algoritmo:ext_rayos}, la extracción de \textit{rayos}, se utilizan dos estructuras llamadas \textit{región de soporte} y \textit{ventana de búsqueda}, estas son de tamaño $L$x$L$ y $(2L+1)$x$(2L+1)$ respectivamente. El valor de la variable $L$ es muy importante ya que define el tamaño de ambas regiones, y a su vez permite definir el tamaño de la región donde se busca el movimiento de los píxeles, por lo cual un gran tamaño puede indicar mayor precisión y a su vez menor velocidad. La idea de estas pruebas es poder encontrar el valor óptimo de $L$ el cual permita tener una respuesta aceptable por parte del algoritmo. 

\subsubsection{Pruebas de normalización de rayos}	

La normalización de rayos, explicada en la Sección~\ref{algoritmo:normalizacion}, es otro de los procesos a los cuales se debe encontrar un valor óptimo a su variable $N$, esta variable indica cual debe ser el largo de los vectores que representan los \textit{rayos} de los vídeos. El proceso de normalización es un proceso en el cual se requiere llevar todos los vectores al mismo espacio vectorial, para así poder realizar comparaciones entre estos. Este proceso tiene una gran desventaja, esto debido a que al transformar un vector de tamaño mayor a $N$ este tiende a perder información en su compresión, al igual que al agrandar un vector de tamaño menor a $N$ se tiende a agregar información que posiblemente sea falsa o simplemente ruido.

\subsubsection{Pruebas del tamaño del \textit{Bag of Visual Words}}
La técnica de \textit{Bag of Visual Words} utiliza la variable $K$, está representa el numero de \textit{clusters} o grupos que se deben formar en el proceso de creación de la bolsa de palabras, a su vez también indica el tamaño del descriptor final o macro-descriptor que representa a cada vídeo. Esta variable es la  más importante en el algoritmo, esto debido a que una mala elección de la cantidad de \textit{clusters}, puede desencadenar en una mala creación de los macro-descriptores, por lo cual la búsqueda del valor óptimo es muy importante para obtener una buena precisión. La única forma de poder encontrar un valor óptimo o funcional es a través del método de prueba y error, esto debido a que no existe una forma empírica de demostrar cual es el valor óptimo para cada corrida.



\subsection{Pruebas codificando los vídeos}
\label{exp:cod}
En la sección~\ref{sec:micro_descriptores}, extracción de micro-descriptores, se explica que antes de poder obtener los \textit{rayos} es necesario realizar un proceso de codificación sobre las imágenes, esto pudiendo ayudar a la extracción de los descriptores. Para esto se preparan tres pruebas distintas, en las cuales se utilizan distintas técnicas de codificación.

	\subsubsection{Sin codificación}
	Esta prueba consiste en ver la eficiencia del descriptor encontrado al final del proceso del algoritmo sin utilizar ningún tipo de codificación sobre los valores de los píxeles, por lo cual el proceso de extracción de rayos se realiza sobre el valor de la intensidad de cada imagen.

	\subsubsection{Codificación de LBP}
	Esta prueba consiste en ver la eficacia del descriptor encontrado al final del proceso del algoritmo propuesto, utilizando el proceso de codificación LBP explicado en la Sección~\ref{sec:lbp}. Esta codificación permite que los nuevos valores obtenidos para los píxeles estén relacionados directamente con la vecindad mas cercana, por lo cual esto permitiría poder realizar una mejor extracción de rayos.


\chapter[Conclusión]{Conclusión}
\label{ch:conclusion}

El método propuesto en este trabajo, nace de la idea de poder realizar el reconocimiento de expresiones faciales en secuencias de imágenes. La idea es proponer una solución simple y a la vez robusta que permita una fácil implementación en distintos ambientes. Principalmente es poder realizar un seguimiento de las regiones que aporten mayor información sobre las expresiones, como la nariz, la boca, los ojos, \etc. este seguimiento lo llamamos \textit{rayo de flujo}, el cual es una codificación del movimiento de las regiones a través del tiempo.

A pesar de no tener buenos resultados a la hora de clasificar nuevas expresiones faciales entrantes, con los experimentos expuestos en el Capítulo~\ref{ch:exp_result} logramos demostrar que la teoría sobre el modelado del movimiento de los pixeles utilizando los \textit{rayos de flujo} puede ser utilizada para realizar el seguimiento de macro-patrones en el rostro, y todo esto contrastándolo con los objetivos podemos concluir:

El nuevo descriptor espacio temporal introducido en esta investigación, llamado \textit{rayo de flujo}, realiza un modelado aproximado del movimiento de los píxeles en el rostro, revisando los resultados obtenidos en la Sección~\ref{exp:micro-descriptores}, logramos deducir que existe una aproximación muy cercana a la hora de modelar los movimientos verticales del rostro humano, por el contrario al tratar de modelar los movimientos horizontales se producía un error considerable. También observamos que mientras más grande es la región de soporte, mayor será la probabilidad de modelar el movimiento de los pixeles a través del video, con esto también nos dimos cuenta que las texturas de las regiones observadas son de vital importancia, puesto que las texturas más definidas como los ojos, la boca o la nariz tienen una muy baja probabilidad de confusión con otras regiones, no así las regiones con texturas mas planas como las mejillas o la frente.

Para poder comparar un \textit{rayos} con otro de un video distinto se tuvo que crear una transformación que permitiera mapear todos los rayos a un mismo espacio o en simples palabras, que todos los rayos tengan el mismo largo. Para realizar esto utilizamos una razón de normalización dependiente de la variable $N$, la cual permitió transformar los \textit{rayos de soporte}. En general el comportamiento de esta variable era estable para diferentes valores utilizados, cabe destacar que realizar una ampliación o reducción del tamaño de los \textit{rayos}, puede provocar una adición de ruido o  una perdida de información. 

Al construir los macro-descriptores, logramos observar que el agrupamiento formado por estos, permitió realizar un estudio sobre los comportamientos de los movimientos de los píxeles sobre el rostro, esto puede ser de gran ayuda para investigaciones futuras sobre el tema, debido a que estas distribuciones pueden tener patrones de movimiento de las partes importantes del rostro humano, que aun no han sido descubiertos por los científicos que estudian este tema, ademas, estos movimientos ya localizados por la etapa el \textit{clustering}, pueden servir para variantes de este método, para encontrar los movimientos de los FACS propuestos por Ekman.

\section{Trabajos Futuros}

A continuación se describen algunos trabajos de investigación que se pueden realizar para lograr obtener mejores resultados:

\begin{itemize}
	
	\item Con respecto a la forma matemática de modelar el \textit{rayo de flujo}, creemos es es necesario investigar el comportamiento de la predicción cambiando la forma como se mapea en el vector los movimientos, en esta investigación el vector representante contenía un par que representante de la velocidad de movimiento en $x$ y $y$, creemos que los resultados obtenidos pueden ser mejorados al utilizar solo la componente $y$, como fue visto en los experimentos, el modelado de esta variable obtenía en todos los casos un error muy pequeño y mucho mas estable que su contraparte. Otra posible modificación es que en vez de utilizar un par con las velocidad, seria mejor calcular el angulo de movimiento para cada instancia del \textit{rayo}.
	
	\item A la hora de realizar el \textit{clustering}, observamos que los resultados no fueron satisfactorios a la hora de realizar la clasificación de las expresiones, pero los resultados obtenidos al modelar los agrupamientos sobre los rostros nos dio ideas para investigar si es posible calcular estos movimientos de forma localizada y encontrar cuales son los movimientos representantes de cada expresión en el rostro. También incitamos a investigar sobre las FACS propuestas por Ekman, por lo que creemos puede ser posible realizar una asociacion de los movimientos del rostro y las técnicas que utilizan para el reconocimiento de las expresiones.
	
	\item Como ultimo trabajo futuro proponemos realizar una segmentación de las partes importantes del rostro a la hora de realizar la extracción de \textit{rayos}, esto debido a que la gran mayoría de los \textit{clusters} importantes estaban ubicados en estas partes. Esto permitirá eliminar mucho ruido producido por \textit{rayos} de regiones sin importancia el rostro.
	
\end{itemize}

%\input{cap-tema}
% incluya otros archivos según su necesidad
% \input{archivo}


% Iniciamos el resto de secciones adicionales al contenido: referencias y apendices
\backmatter


% Bibliografía
% referencias.bib es el archivo con la base de datos bibliografica
% se recomienda utilizar un manejador de referencias: Jabref (jabref.sourceforge.net)
% El estilo por defecto es IEEE Transactions
\bibliographystyle{ieeetr}
% Acá puede incluir uno más archivos de referencia
\bibliography{referencias}


% Simbología y glosario
% Utilice un paquete para generar símbolos y glosarios.
% Por ejemplo: nomencl (http://texdoc.net/pkg/nomencl)


% Anexos
%\appendix

% Aca se incluyen los archivos con el texto de los anexos
% Por ejemplo, anexo.tex
%\chapter{Tablas de resultados}
\label{ch:anexo-a}

\begin{table}[tb]
	\centering
	\pgfplotstabletypeset[tablaSVMresults]{Datos/resultados_generales.dat}
	\caption{Resultados obtenidos utilizando las mejores variables obtenidas en el proceso de experimentos.}
	\label{tabla:exp:accuracy_general}
\end{table}



\chapter{Segundo anexo}
\label{ch:anexo-b}


%\blindtext[10]



% puede incluir más archivos de anexos
% \input{anexo-dos}

\end{document}
% that's all folks