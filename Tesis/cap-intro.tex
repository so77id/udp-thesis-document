\chapter[Introducción]{Introducción}
\label{ch:intro}

\section{Motivación}
\label{sec:motivacion}
Con los actuales avances tecnológicos, el reconocimiento de expresiones faciales o FER (por sus siglas en inglés) forma parte importante de una rama de la inteligencia artificial, específicamente el reconocimiento de patrones.  Las investigaciones sobre el FER permiten el desarrollo de futuras tecnologías ligadas al desarrollo de interfaces centradas en el humano. 

Existen diversas aplicaciones tecnológicas enfocadas al reconocimiento de expresiones faciales aplicadas en distintas áreas del conocimiento humano, tales como:  La medicina, que se utiliza para la detección de enfermedades y rehabilitación de trastornos mentales; la psicología, para reportes sobre estados de ánimo; el marketing, la posibilidad de una segmentación del cliente al cual se quiere llegar con la publicidad; la robótica, la interacción humano-robot; la seguridad, para combatir fraudes de pasaportes, soporte al orden público e identificación de personas desaparecidas; entre otras áreas.

Diversas empresas también han utilizado estos avances de reconocimiento facial como un componente que aporta información en las tecnologías que ofrecen. Un ejemplo de esto es  Google que utiliza el FER en su motor de búsqueda de imágenes y se proyecta utilizar este reconocimiento para aportar información a la entrada de vídeo de los Google Glass. Otro ejemplo es Facebook que usa la detección de rostros en  fotos para realizar el etiquetado rápido de usuarios.

Los estudios de Mehrabian~\cite{MEHRABIAN1968} sobre la comunicación oral indican que las expresiones faciales contribuyen con aproximadamente el 55\% de la transmisión del mensaje, lo cual es un valor mucho mayor que la parte verbal con 7\% y la parte vocal un 38\%.

El objetivo de este proyecto es buscar un nuevo descriptor espacio temporal para reconocer expresiones faciales. Para esto se propone un nuevo método utilizando imágenes dinámicas, el cual utilizaremos para reconocer las seis expresiones universales tales como: disgusto, cólera, felicidad, tristeza, miedo y sorpresa.

\section{Problema}
\label{sec:problema}



\section{Solución propuesta}
\label{sec:solucion}

Para este nuevo método se introduce un nuevo micro-descriptor basado en técnicas similares al optical flow, el cual nos permite realizar el seguimiento de las regiones de interés o ROI (por sus siglas en inglés) que denominaremos ``rayo de flujo'' o simplemente ``rayo''.
La cadena de procedimientos ordenados o ``pipeline'' de este método se divide en cuatro grandes
etapas: preprocesamiento de la base de datos, extracción de micro-descriptores, creación de macro-descriptores y clasificación.

\textbf{Preprocesamiento de la base de datos.}
\label{ch1:par:preprocesamientobdd}
Antes de poder realizar la extracción de características de los vídeos, es necesario realizar un preprocesamiento sobre éstos, esto debido a que los vídeos pueden tener ciertos elementos que estropean la extracción de características. Para evitar agregar ruido al vídeo se utilizan dos técnicas que permiten obtener solo el rostro de la persona a representar su expresión y a su vez corregir los movimientos que éstos puedan tener. Esta etapa sera profundizada en la Sección~\ref{sec:proc_bdd}.


\textbf{Extracción de micro-descriptores.}
\label{ch1:par:microdescriptores}
Éste proceso consiste en la extracción de los rayos de flujo de cada uno de los vídeos preprocesados, para esto, primero se seleccionan las regiones de interés de la cara. Luego, para cada uno de los píxeles de la región, se procede a calcular el movimiento de este píxel con respecto al cuadro siguiente, y así sucesivamente con cada uno de los cuadros del vídeo. Luego de calcular el movimiento para cada píxel en cada cuadro, se obtiene el conjunto de rayos o micro-descriptores que definen el vídeo. Esta etapa sera profundizada en la Sección~\ref{sec:micro-descriptores}.

\textbf{Creación de macro-descriptores.}
\label{ch1:par:macrodescriptores}
Luego de extraer cada uno de los micro-descriptores, se procede a crear utilizar la técnica de la bolsa de palabras (bag of words por sus siglas en ingles), este método es utilizado para poder agrupar los rayos en $k$ grupos, los cuales definen el espacio de los macro-descriptores. Cada rayo extraído de los vídeos pertenece únicamente a un solo grupo, por lo tanto es posible para cada vídeo obtener un histograma de la frecuencia de aparición de cada uno de estos grupos en sus rayos. Este histograma obtenido es el macro-descriptor que se utiliza para poder entrenar y clasificar. Esta etapa sera profundizada en la Sección~\ref{sec:macro-descriptores}.

\textbf{Entrenamiento y clasificación.}
\label{ch1:par:entrnamientoyclasificacion}
Luego de obtener los los macro-des\-crip\-tores para cada uno de los vídeos, se procede a la etapa de entrenamiento y posterior clasificación. Para poder entrenar y clasificar con los macro-descriptores obtenidos en la etapa anterior, se utilizan las Maquinas de Vectores de Soporte (Support Vector Machines por sus siglas en ingles). Esta técnica permite dividir el espacio vectorial representado por los descriptores de entrenamiento, para luego ser probado por los descriptores que se obtienen en la clasificación o prueba del modelo creado. Para poder probar la efectividad de la clasificación se utiliza la técnica de validación cruzada, la cual permite realizar distintas pruebas de entrenamiento y clasificación, para obtener cual es el porcentaje de efectividad del método. Esta etapa sera profundizada en la Sección~\ref{sec:clasificacion}.


\section{Objetivos}
\label{subsec:objetivos}
A continuación se describen los objetivos de este trabajo:

\subsection{Objetivo general}
\label{subsubsec:objgeneral}
Crear un descriptor para el reconocimiento de expresiones faciales en vídeo utilizando micro patrones basados en el seguimiento del flujo de los movimientos del rostro.

\subsection{Objetivos específicos}
\label{subsubsec:objgeneral}
	\begin{enumerate}
		\item Definir un método de elección de las regiones de interés del rostro.
		\item Definir una codificación que permita la normalización de los rayos de flujo.
		\item Crear o encontrar una métrica que permita la comparación entre rayos luego del proceso de normalización.
		\item Utilizar y probar distintos algoritmos de clasificación.
		\item Evaluar y comparar el método creado con el estado del arte hasta antes de comenzar la memoria. 
	\end{enumerate}