\chapter[Conclusión]{Conclusión}
\label{ch:conclusion}

El método propuesto en este trabajo, nace de la idea de poder realizar el reconocimiento de expresiones faciales en secuencias de imágenes. La idea es proponer una solución simple y a la vez robusta que permita una fácil implementación en distintos ambientes. Principalmente es poder realizar un seguimiento de las regiones que aporten mayor información sobre las expresiones, como la nariz, la boca, los ojos, \etc. este seguimiento lo llamamos \textit{rayo de flujo}, el cual es una codificación del movimiento de las regiones a través del tiempo.

A pesar de no tener buenos resultados a la hora de clasificar nuevas expresiones faciales entrantes, con los experimentos expuestos en el Capítulo~\ref{ch:exp_result} logramos demostrar que la teoría sobre el modelado del movimiento de los pixeles utilizando los \textit{rayos de flujo} puede ser utilizada para realizar el seguimiento de macro-patrones en el rostro, lo cual nos permitió responder las preguntas planteadas en la Sección~\ref{intro:ventajas}: ¿Permitirán los \textit{rayos} realizar un modelado espacio-temporal?,¿permitirán ver cual es el comportamiento de los pixeles a través del tiempo?, por lo visto en el capítulo de resultados, los \textit{rayos de flujo} permiten realizar un modelado espacio-temporal de cada uno de los pixeles a traves del video, lo cual nos permite hacer un seguimiento de estos movimientos y realizar una codificacion que nos permita crear representaciones vectoriales de los videos;  ¿Permitirán los \textit{rayos} modelar micro-patrones en los movimientos del rostro que no pueden ser vistos a simple vista por los humanos?, de ser cierto, ¿estos micro-patrones aportarán mayor información al modelo de clasificación?, el modelado de micro-patrones no queda muy claro al revisar el comportamiento de los pixeles a lo largo del tiempo, en general se ve una tendencia a de ciertos movimientos localizados en las regiones importantes del rostro, dichos movimientos pueden ser visualizados en la Tabla~\ref{tabla:comparacion_clusters}, por lo cual proponemos seguir investigando esta área; ¿Existe la posibilidad de que cada una de las expresiones faciales universales pueda tener asociado un conjunto de \textit{rayos} que la definan?, no descartamos la posibilidad de que la existencia de micro-patrones o movimientos de regiones que definan un expresión, por lo cual es necesario realizar una segmentación de las partes importantes del rostro y realizar el procesamiento solo con esas partes, lo cual reducirá de forma considerable el ruido generado por las partes que no aportan información. Ademas podemos concluir:

El nuevo descriptor espacio temporal introducido en esta investigación, llamado \textit{rayo de flujo}, realiza un modelado aproximado del movimiento de los píxeles en el rostro, revisando los resultados obtenidos en la Sección~\ref{exp:micro-descriptores}, logramos deducir que existe una aproximación muy cercana a la hora de modelar los movimientos verticales del rostro humano, por el contrario al tratar de modelar los movimientos horizontales se producía un error considerable. También observamos que mientras más grande es la región de soporte, mayor será la probabilidad de modelar el movimiento de los pixeles a través del video, con esto también nos dimos cuenta que las texturas de las regiones observadas son de vital importancia, puesto que las texturas más definidas como los ojos, la boca o la nariz tienen una muy baja probabilidad de confusión con otras regiones, no así las regiones con texturas mas planas como las mejillas o la frente.

Para poder comparar un \textit{rayos} con otro de un video distinto se tuvo que crear una transformación que permitiera mapear todos los rayos a un mismo espacio o en simples palabras, que todos los rayos tengan el mismo largo. Para realizar esto utilizamos una razón de normalización dependiente de la variable $N$, la cual permitió transformar los \textit{rayos de soporte}. En general el comportamiento de esta variable era estable para diferentes valores utilizados, cabe destacar que realizar una ampliación o reducción del tamaño de los \textit{rayos}, puede provocar una adición de ruido o  una perdida de información. 

Al construir los macro-descriptores, logramos observar que el agrupamiento formado por estos, permitió realizar un estudio sobre los comportamientos de los movimientos de los píxeles sobre el rostro, esto puede ser de gran ayuda para investigaciones futuras sobre el tema, debido a que estas distribuciones pueden tener patrones de movimiento de las partes importantes del rostro humano, que aun no han sido descubiertos por los científicos que estudian este tema, además, estos movimientos ya localizados por la etapa el \textit{clustering}, pueden servir para variantes de este método, para encontrar los movimientos de los FACS propuestos por Ekman.

\section{Trabajos Futuros}

A continuación se describen algunos trabajos de investigación que se pueden realizar para lograr obtener mejores resultados:

\begin{itemize}
	
	\item Con respecto a la forma matemática de modelar el \textit{rayo de flujo}, creemos es es necesario investigar el comportamiento de la predicción cambiando la forma como se mapea en el vector los movimientos, en esta investigación el vector representante contenía un par que representaba el movimiento en $x$ y $y$, creemos que los resultados obtenidos pueden ser mejorados al utilizar solo la componente $y$ para modelar el \textit{rayo}, esto debido a que el modelado de esta variable obtenía en todos los casos un error muy pequeño y mucho mas estable que su contraparte. Otra posible modificación es que en vez de utilizar un par con los movimientos, seria mejor calcular el angulo de movimiento para cada instancia del \textit{rayo}.
	
	\item A la hora de realizar el \textit{clustering}, observamos que los resultados no fueron satisfactorios a la hora de realizar la clasificación de las expresiones, pero los resultados obtenidos al modelar los agrupamientos sobre los rostros nos dio ideas para investigar si es posible calcular estos movimientos de forma localizada y encontrar cuales son los movimientos representantes de cada expresión en el rostro. También incitamos a investigar sobre las FACS propuestas por Ekman, por lo que creemos puede ser posible realizar una asociacion de los movimientos del rostro y las técnicas que utilizan para el reconocimiento de las expresiones.
	
	\item Como último trabajo futuro proponemos que antes de realizar la extracción de \textit{rayos}, es necesario hacer una segmentación de las partes importantes del rostro (ojos, nariz y boca), esto debido a que la gran mayoría de los \textit{clusters} importantes estaban ubicados en estas partes. Esto permitirá eliminar mucho ruido producido por \textit{rayos} de regiones sin importancia el rostro.
	
\end{itemize}