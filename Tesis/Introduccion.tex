\chapter[Introducción]{Introducción}\label{ch:capitulo1}
\fpar

\parindent=0pt Lorem ipsum ad his scripta blandit partiendo, eum fastidii accumsan euripidis in, eum liber hendrerit an. 

\vspace{0.5cm}
\parindent=30pt Quo mundi lobortis reformidans eu, legimus senserit definiebas an eos. Eu sit tincidunt incorrupte definitionem, vis mutat affert percipit cu, eirmod consectetuer signiferumque eu per. In usu latine equidem dolores.


%=================================================ANTECEDENTES Y MOTIVACION========================================%

\section{Título del Subcapítulo 1 }\label{chsub:Título del Subcapítulo 1}

\parindent=0pt Quo no falli viris intellegam, ut fugit veritus placerat per. Ius id vidit volumus mandamus, vide veritus democritum te nec, ei eos debet libris consulatu. No mei ferri graeco dicunt, ad cum veri accommodare. 


\subsection{Título de la sección 1 del subcapítulo 1}\label{chsub:Título de la sección 1 del subcapítulo 1}


\parindent=0pt Eos vocibus deserunt quaestio ei. Blandit incorrupte quaerendum in quo, nibh impedit id vis, vel no nullam semper audiam. 

\vspace{0.5cm}
\parindent=30pt Ei populo graeci consulatu mei, has ea stet modus phaedrum. Inani oblique ne has, duo et veritus detraxit. 


\subsubsection{Título de la subsección 1(no aparece en tabla de contenidos)}\label{chsub:Título de la subsección 1}

\parindent=0pt Lorem ipsum ad his scripta blandit partiendo, eum fastidii accumsan euripidis in, eum liber hendrerit an. 



\begin{description}
\item[a)] Punteo de primer nivel 1
\item[b)] Punteo de primer nivel 2

  	\begin{description}
  	\item[i)] Punteo de segundo nivel 1
  	\item[ii)] Punteo de segundo nivel 2
  
  		\begin{description}
  		\item[-] Punteo de tercer nivel 1
  		\item[-] Punteo de tercer nivel 2
		\end{description}  
		
  	\item[iii)] Punteo de segundo nivel 3
	\end{description}

\item[c)] Punteo de primer nivel 3
\end{description}

\begin{enumerate}
\item Enumerar 1.
\item Enumerar 2.
\item Enumerar 3
\end{enumerate}

\begin{itemize}
\item Ítem de lista 1.
\item Ítem de lista 2.
\item Ítem de lista 3.
\end{itemize}


\vspace{0.5cm}
\parindent=30pt Quo mundi lobortis reformidans eu, legimus senserit definiebas an eos. Eu sit tincidunt incorrupte definitionem, vis mutat affert percipit cu, eirmod consectetuer signiferumque eu per. In usu latine equidem dolores.

\subsection{Título de la sección 2 del subcapítulo 1}\label{chsub:Título de la sección 2 del subcapítulo 1}

\parindent=0pt Quo no falli viris intellegam, ut fugit veritus placerat per. Ius id vidit volumus mandamus, vide veritus democritum te nec, ei eos debet libris consulatu. No mei ferri graeco dicunt, ad cum veri accommodare. 
 


\section{Título del Subcapítulo 2} \label{chsub:Título del Subcapítulo 2}

\parindent=0pt Lorem ipsum ad his scripta blandit partiendo, eum fastidii accumsan euripidis in, eum liber hendrerit an. 

\vspace{0.5cm}
\parindent=30pt Quo mundi lobortis reformidans eu, legimus senserit definiebas an eos. Eu sit tincidunt incorrupte definitionem, vis mutat affert percipit cu, eirmod consectetuer signiferumque eu per.

\subsection{Título de la sección 1 del subcapítulo 2}\label{chsub:Título de la sección 1 del subcapítulo 2}

\parindent=0pt Eos vocibus deserunt quaestio ei. Blandit incorrupte quaerendum in quo, nibh impedit id vis, vel no nullam semper audiam. 

