%
% ---------------------------------------------------------------
% template para escribir tesis/memorias en
% la Universidad Diego Portales
% ---------------------------------------------------------------
%
\documentclass[thesis,final]{udpbook}
%
\usepackage[latin1]{inputenc}    % esto NO es portable

\usepackage[T1]{fontenc}
\usepackage{amsmath}

\usepackage{amsmath} %%% para split
\usepackage{makeidx,color}  % allows for indexgeneration
\usepackage{graphicx}

\usepackage{tabulary}

\usepackage{fixltx2e}

\usepackage{listings}
\usepackage{color}
\usepackage{rotating}
\usepackage{longtable}
\usepackage{float}


%\usepackage{fancyvrb}
%\DefineVerbatimEnvironment{code}{Verbatim}{fontsize=\small}
%\DefineVerbatimEnvironment{example}{Verbatim}{fontsize=\small}



%\providecommand{\abs}[1]{\lvert#1\rvert} 
%\providecommand{\norm}[1]{\lVert#1\rVert} 


%
% ----------------------------------------------------------------
% defina su escuela
%
\udpEscuela{Escuela Ingenier�a Inform�tica}{Facultad de Ingenier�a}
%
% ---------------------------------------------------------------
% Comienzo del documento
% ---------------------------------------------------------------
\usepackage{hyperref}
\hypersetup{%
    pdfborder = {0 0 0}
}
\usepackage[numbered]{bookmark}
\begin{document}
\frontmatter
%
% ----------------------------------------------------------------
% pagina de titulo
% ----------------------------------------------------------------
%
\begin{titlepage}
\title{T�tulo de tesis o memoria}
\author{Nombre del autor}
\degreedoc{Tesis/Memoria}{grado/t�tulo}{Ingeniero Civil en Inform�tica y Telecomunicaciones}
\chairperson{PROFESOR GU�A}      %
\committee{PROFESOR COMIT� 1 \\ PROFESOR COMIT� 2}     % separados por \\
\date{(MES), (A�O)}	% (Ej: Enero, 2006)
\end{titlepage}                      %
%
% ----------------------------------------------------------------
% pagina de firmas
% ----------------------------------------------------------------
% (maximo seis firmas)
%
\begin{signaturepage}                   %
\approval{NOMBRE\\Profesor gu�a}        %
\approvaldouble{NOMBRE\\Comit�}{NOMBRE\\Comit�}%
\end{signaturepage}                     %
%
% ----------------------------------------------------------------
% pagina dedicatoria
% ----------------------------------------------------------------
% separadas por \\ si es mas de una linea
%
\begin{dedicatory}                      %
Una peque�a dedicatoria.                %
\end{dedicatory}                        %
%
% ----------------------------------------------------------------
% Indices de materia, figuras y tablas
% ----------------------------------------------------------------
% paginas de contenido (indice de materias), lista de figuras
% y lista de tablas.
%
\tableofcontents                        % tabla de contenido
\listoffigures                          % �ndice de figuras
\listoftables                           % �ndice de tablas
%
% ----------------------------------------------------------------
% pagina de agradecimientos
% ----------------------------------------------------------------
%
\begin{acknowledgment}                  %
(Nota redactada sobriamente en la cual se agradece a quienes han colaborado en la elaboraci�n del trabajo.)
\end{acknowledgment}                    %
%
% ----------------------------------------------------------------
% pagina de abstract
% ----------------------------------------------------------------
%
\begin{abstract}                        %
(Abstract es el resumen en ingl�s que no debe exceder una p�gina.)
\end{abstract}                          %
%
% ----------------------------------------------------------------
% pagina de resumen
% ----------------------------------------------------------------
%
\begin{resumen}                         %
(El resumen no debe contener menos de 100 palabras ni mas de 300 palabras.)
\end{resumen}                           %
%
% ----------------------------------------------------------------
% Fin de las paginas iniciales
% ----------------------------------------------------------------
%
\cleardoublepage                        %
\mainmatter                             %
                                        %
% ----------------------------------------------------------------
% Cap�tulos y secciones del documento
% ----------------------------------------------------------------
% aca se incluyen los archivos con el texto de los capitulos
% (Ej.: cha-intro.tex es el archivo con un capitulo)
%
\input{CapI}

% ... mas archivos de capitulos
%
% ---------------------------------------------------------------
% Bibliograf�a
% ---------------------------------------------------------------
% tubiblio.bib es el archivo con la base de datos bibliografica
%
\bibliographystyle{ieeetr}
\bibliography{referencias}
%
% ---------------------------------------------------------------
% Simbolog�a y glosario
% ---------------------------------------------------------------
% simbolos.tex es el archivo de simbolos (y glosario)
%
%\begin{symbology}
%\input{simbolos}  % archivo propio de simbolos
%\end{symbology}
%
% ---------------------------------------------------------------
% Anexos
% ---------------------------------------------------------------
\appendix
%
% aca se incluyen los archivos con el texto de los anexos
% (Ej.: anx-uno.tex es el archivo de un anexo)
%
% ... mas archivos de anexos
%
% ---------------------------------------------------------------
% Fin del documento
% NO ESCRIBIR DESPU�S DE ESTA LINEA
\backmatter
\end{document}
% ---------------------------------------------------------------
