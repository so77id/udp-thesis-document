%
% ---------------------------------------------------------------
% template para escribir tesis/memorias en
% la Universidad Diego Portales
% ---------------------------------------------------------------
%
\documentclass[thesis,final]{udpbook}
%
\usepackage[latin1]{inputenc}    % esto NO es portable

\usepackage[T1]{fontenc}
\usepackage{amsmath}

\usepackage{amsmath} %%% para split
\usepackage{makeidx,color}  % allows for indexgeneration
\usepackage{graphicx}

\usepackage{tabulary}

\usepackage{fixltx2e}

\usepackage{listings}
\usepackage{color}
\usepackage{rotating}
\usepackage{longtable}
\usepackage{float}


%\usepackage{fancyvrb}
%\DefineVerbatimEnvironment{code}{Verbatim}{fontsize=\small}
%\DefineVerbatimEnvironment{example}{Verbatim}{fontsize=\small}



%\providecommand{\abs}[1]{\lvert#1\rvert} 
%\providecommand{\norm}[1]{\lVert#1\rVert} 


%
% ----------------------------------------------------------------
% defina su escuela
%
\udpEscuela{Escuela Ingenier�a Inform�tica}{Facultad de Ingenier�a}
%
% ---------------------------------------------------------------
% Comienzo del documento
% ---------------------------------------------------------------
\usepackage{hyperref}
\hypersetup{%
    pdfborder = {0 0 0}
}
\usepackage[numbered]{bookmark}
\begin{document}
\frontmatter
%
% ----------------------------------------------------------------
% pagina de titulo
% ----------------------------------------------------------------
%
\begin{titlepage}
\title{T�tulo de tesis o memoria}
\author{Nombre del autor}
\degreedoc{Tesis/Memoria}{grado/t�tulo}{Ingeniero Civil en Inform�tica y Telecomunicaciones}
\chairperson{PROFESOR GU�A}      %
\committee{PROFESOR COMIT� 1 \\ PROFESOR COMIT� 2}     % separados por \\
\date{(MES), (A�O)}	% (Ej: Enero, 2006)
\end{titlepage}                      %
%
% ----------------------------------------------------------------
% pagina de firmas
% ----------------------------------------------------------------
% (maximo seis firmas)
%
\begin{signaturepage}                   %
\approval{NOMBRE\\Profesor gu�a}        %
\approvaldouble{NOMBRE\\Comit�}{NOMBRE\\Comit�}%
\end{signaturepage}                     %
%
% ----------------------------------------------------------------
% pagina dedicatoria
% ----------------------------------------------------------------
% separadas por \\ si es mas de una linea
%
\begin{dedicatory}                      %
Una peque�a dedicatoria.                %
\end{dedicatory}                        %
%
% ----------------------------------------------------------------
% Indices de materia, figuras y tablas
% ----------------------------------------------------------------
% paginas de contenido (indice de materias), lista de figuras
% y lista de tablas.
%
\tableofcontents                        % tabla de contenido
\listoffigures                          % �ndice de figuras
\listoftables                           % �ndice de tablas
%
% ----------------------------------------------------------------
% pagina de agradecimientos
% ----------------------------------------------------------------
%
\begin{acknowledgment}                  %
(Nota redactada sobriamente en la cual se agradece a quienes han colaborado en la elaboraci�n del trabajo.)
\end{acknowledgment}                    %
%
% ----------------------------------------------------------------
% pagina de abstract
% ----------------------------------------------------------------
%
\begin{abstract}                        %
(Abstract es el resumen en ingl�s que no debe exceder una p�gina.)
\end{abstract}                          %
%
% ----------------------------------------------------------------
% pagina de resumen
% ----------------------------------------------------------------
%
\begin{resumen}                         %
(El resumen no debe contener menos de 100 palabras ni mas de 300 palabras.)
\end{resumen}                           %
%
% ----------------------------------------------------------------
% Fin de las paginas iniciales
% ----------------------------------------------------------------
%
\cleardoublepage                        %
\mainmatter                             %
                                        %
% ----------------------------------------------------------------
% Cap�tulos y secciones del documento
% ----------------------------------------------------------------
% aca se incluyen los archivos con el texto de los capitulos
% (Ej.: cha-intro.tex es el archivo con un capitulo)
%
\chapter[T�TULO DEL PRIMER CAP�TULO]{T�TULO DEL PRIMER CAP�TULO}\label{ch:capitulo1}
\fpar

\parindent=0pt Lorem ipsum ad his scripta blandit partiendo, eum fastidii accumsan euripidis in, eum liber hendrerit an. 

\vspace{0.5cm}
\parindent=30pt Quo mundi lobortis reformidans eu, legimus senserit definiebas an eos. Eu sit tincidunt incorrupte definitionem, vis mutat affert percipit cu, eirmod consectetuer signiferumque eu per. In usu latine equidem dolores.


%=================================================ANTECEDENTES Y MOTIVACION========================================%

\section{T�tulo del Subcap�tulo 1 }\label{chsub:T�tulo del Subcap�tulo 1}

\parindent=0pt Quo no falli viris intellegam, ut fugit veritus placerat per. Ius id vidit volumus mandamus, vide veritus democritum te nec, ei eos debet libris consulatu. No mei ferri graeco dicunt, ad cum veri accommodare. 


\subsection{T�tulo de la secci�n 1 del subcap�tulo 1}\label{chsub:T�tulo de la secci�n 1 del subcap�tulo 1}


\parindent=0pt Eos vocibus deserunt quaestio ei. Blandit incorrupte quaerendum in quo, nibh impedit id vis, vel no nullam semper audiam. 

\vspace{0.5cm}
\parindent=30pt Ei populo graeci consulatu mei, has ea stet modus phaedrum. Inani oblique ne has, duo et veritus detraxit. 


\subsubsection{T�tulo de la subsecci�n 1(no aparece en tabla de contenidos)}\label{chsub:T�tulo de la subsecci�n 1}

\parindent=0pt Lorem ipsum ad his scripta blandit partiendo, eum fastidii accumsan euripidis in, eum liber hendrerit an. 



\begin{description}
\item[a)] Punteo de primer nivel 1
\item[b)] Punteo de primer nivel 2

  	\begin{description}
  	\item[i)] Punteo de segundo nivel 1
  	\item[ii)] Punteo de segundo nivel 2
  
  		\begin{description}
  		\item[-] Punteo de tercer nivel 1
  		\item[-] Punteo de tercer nivel 2
		\end{description}  
		
  	\item[iii)] Punteo de segundo nivel 3
	\end{description}

\item[c)] Punteo de primer nivel 3
\end{description}

\begin{enumerate}
\item Enumerar 1.
\item Enumerar 2.
\item Enumerar 3
\end{enumerate}

\begin{itemize}
\item �tem de lista 1.
\item �tem de lista 2.
\item �tem de lista 3.
\end{itemize}


\vspace{0.5cm}
\parindent=30pt Quo mundi lobortis reformidans eu, legimus senserit definiebas an eos. Eu sit tincidunt incorrupte definitionem, vis mutat affert percipit cu, eirmod consectetuer signiferumque eu per. In usu latine equidem dolores.

\subsection{T�tulo de la secci�n 2 del subcap�tulo 1}\label{chsub:T�tulo de la secci�n 2 del subcap�tulo 1}

\parindent=0pt Quo no falli viris intellegam, ut fugit veritus placerat per. Ius id vidit volumus mandamus, vide veritus democritum te nec, ei eos debet libris consulatu. No mei ferri graeco dicunt, ad cum veri accommodare. 
 


\section{T�tulo del Subcap�tulo 2} \label{chsub:T�tulo del Subcap�tulo 2}

\parindent=0pt Lorem ipsum ad his scripta blandit partiendo, eum fastidii accumsan euripidis in, eum liber hendrerit an. 

\vspace{0.5cm}
\parindent=30pt Quo mundi lobortis reformidans eu, legimus senserit definiebas an eos. Eu sit tincidunt incorrupte definitionem, vis mutat affert percipit cu, eirmod consectetuer signiferumque eu per.

\subsection{T�tulo de la secci�n 1 del subcap�tulo 2}\label{chsub:T�tulo de la secci�n 1 del subcap�tulo 2}

\parindent=0pt Eos vocibus deserunt quaestio ei. Blandit incorrupte quaerendum in quo, nibh impedit id vis, vel no nullam semper audiam. 



% ... mas archivos de capitulos
%
% ---------------------------------------------------------------
% Bibliograf�a
% ---------------------------------------------------------------
% tubiblio.bib es el archivo con la base de datos bibliografica
%
\bibliographystyle{ieeetr}
\bibliography{referencias}
%
% ---------------------------------------------------------------
% Simbolog�a y glosario
% ---------------------------------------------------------------
% simbolos.tex es el archivo de simbolos (y glosario)
%
%\begin{symbology}
%\input{simbolos}  % archivo propio de simbolos
%\end{symbology}
%
% ---------------------------------------------------------------
% Anexos
% ---------------------------------------------------------------
\appendix
%
% aca se incluyen los archivos con el texto de los anexos
% (Ej.: anx-uno.tex es el archivo de un anexo)
%
% ... mas archivos de anexos
%
% ---------------------------------------------------------------
% Fin del documento
% NO ESCRIBIR DESPU�S DE ESTA LINEA
\backmatter
\end{document}
% ---------------------------------------------------------------
